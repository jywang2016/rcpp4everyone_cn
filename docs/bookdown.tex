\documentclass[]{ctexbook}
\usepackage{lmodern}
\usepackage{amssymb,amsmath}
\usepackage{ifxetex,ifluatex}
\usepackage{fixltx2e} % provides \textsubscript
\ifnum 0\ifxetex 1\fi\ifluatex 1\fi=0 % if pdftex
  \usepackage[T1]{fontenc}
  \usepackage[utf8]{inputenc}
\else % if luatex or xelatex
  \ifxetex
    \usepackage{xltxtra,xunicode}
  \else
    \usepackage{fontspec}
  \fi
  \defaultfontfeatures{Ligatures=TeX,Scale=MatchLowercase}
\fi
% use upquote if available, for straight quotes in verbatim environments
\IfFileExists{upquote.sty}{\usepackage{upquote}}{}
% use microtype if available
\IfFileExists{microtype.sty}{%
\usepackage{microtype}
\UseMicrotypeSet[protrusion]{basicmath} % disable protrusion for tt fonts
}{}
\usepackage[b5paper,tmargin=2.5cm,bmargin=2.5cm,lmargin=3.5cm,rmargin=2.5cm]{geometry}
\usepackage[unicode=true]{hyperref}
\PassOptionsToPackage{usenames,dvipsnames}{color} % color is loaded by hyperref
\hypersetup{
            pdftitle={老少咸宜Rcpp},
            pdfauthor={Masaki E. Tsuda 著 jywang 译},
            colorlinks=true,
            linkcolor=Maroon,
            citecolor=Blue,
            urlcolor=Blue,
            breaklinks=true}
\urlstyle{same}  % don't use monospace font for urls
\usepackage{natbib}
\bibliographystyle{apalike}
\usepackage{color}
\usepackage{fancyvrb}
\newcommand{\VerbBar}{|}
\newcommand{\VERB}{\Verb[commandchars=\\\{\}]}
\DefineVerbatimEnvironment{Highlighting}{Verbatim}{commandchars=\\\{\}}
% Add ',fontsize=\small' for more characters per line
\usepackage{framed}
\definecolor{shadecolor}{RGB}{248,248,248}
\newenvironment{Shaded}{\begin{snugshade}}{\end{snugshade}}
\newcommand{\KeywordTok}[1]{\textcolor[rgb]{0.13,0.29,0.53}{\textbf{#1}}}
\newcommand{\DataTypeTok}[1]{\textcolor[rgb]{0.13,0.29,0.53}{#1}}
\newcommand{\DecValTok}[1]{\textcolor[rgb]{0.00,0.00,0.81}{#1}}
\newcommand{\BaseNTok}[1]{\textcolor[rgb]{0.00,0.00,0.81}{#1}}
\newcommand{\FloatTok}[1]{\textcolor[rgb]{0.00,0.00,0.81}{#1}}
\newcommand{\ConstantTok}[1]{\textcolor[rgb]{0.00,0.00,0.00}{#1}}
\newcommand{\CharTok}[1]{\textcolor[rgb]{0.31,0.60,0.02}{#1}}
\newcommand{\SpecialCharTok}[1]{\textcolor[rgb]{0.00,0.00,0.00}{#1}}
\newcommand{\StringTok}[1]{\textcolor[rgb]{0.31,0.60,0.02}{#1}}
\newcommand{\VerbatimStringTok}[1]{\textcolor[rgb]{0.31,0.60,0.02}{#1}}
\newcommand{\SpecialStringTok}[1]{\textcolor[rgb]{0.31,0.60,0.02}{#1}}
\newcommand{\ImportTok}[1]{#1}
\newcommand{\CommentTok}[1]{\textcolor[rgb]{0.56,0.35,0.01}{\textit{#1}}}
\newcommand{\DocumentationTok}[1]{\textcolor[rgb]{0.56,0.35,0.01}{\textbf{\textit{#1}}}}
\newcommand{\AnnotationTok}[1]{\textcolor[rgb]{0.56,0.35,0.01}{\textbf{\textit{#1}}}}
\newcommand{\CommentVarTok}[1]{\textcolor[rgb]{0.56,0.35,0.01}{\textbf{\textit{#1}}}}
\newcommand{\OtherTok}[1]{\textcolor[rgb]{0.56,0.35,0.01}{#1}}
\newcommand{\FunctionTok}[1]{\textcolor[rgb]{0.00,0.00,0.00}{#1}}
\newcommand{\VariableTok}[1]{\textcolor[rgb]{0.00,0.00,0.00}{#1}}
\newcommand{\ControlFlowTok}[1]{\textcolor[rgb]{0.13,0.29,0.53}{\textbf{#1}}}
\newcommand{\OperatorTok}[1]{\textcolor[rgb]{0.81,0.36,0.00}{\textbf{#1}}}
\newcommand{\BuiltInTok}[1]{#1}
\newcommand{\ExtensionTok}[1]{#1}
\newcommand{\PreprocessorTok}[1]{\textcolor[rgb]{0.56,0.35,0.01}{\textit{#1}}}
\newcommand{\AttributeTok}[1]{\textcolor[rgb]{0.77,0.63,0.00}{#1}}
\newcommand{\RegionMarkerTok}[1]{#1}
\newcommand{\InformationTok}[1]{\textcolor[rgb]{0.56,0.35,0.01}{\textbf{\textit{#1}}}}
\newcommand{\WarningTok}[1]{\textcolor[rgb]{0.56,0.35,0.01}{\textbf{\textit{#1}}}}
\newcommand{\AlertTok}[1]{\textcolor[rgb]{0.94,0.16,0.16}{#1}}
\newcommand{\ErrorTok}[1]{\textcolor[rgb]{0.64,0.00,0.00}{\textbf{#1}}}
\newcommand{\NormalTok}[1]{#1}
\usepackage{longtable,booktabs}
% Fix footnotes in tables (requires footnote package)
\IfFileExists{footnote.sty}{\usepackage{footnote}\makesavenoteenv{long table}}{}
\IfFileExists{parskip.sty}{%
\usepackage{parskip}
}{% else
\setlength{\parindent}{0pt}
\setlength{\parskip}{6pt plus 2pt minus 1pt}
}
\setlength{\emergencystretch}{3em}  % prevent overfull lines
\providecommand{\tightlist}{%
  \setlength{\itemsep}{0pt}\setlength{\parskip}{0pt}}
\setcounter{secnumdepth}{5}
% Redefines (sub)paragraphs to behave more like sections
\ifx\paragraph\undefined\else
\let\oldparagraph\paragraph
\renewcommand{\paragraph}[1]{\oldparagraph{#1}\mbox{}}
\fi
\ifx\subparagraph\undefined\else
\let\oldsubparagraph\subparagraph
\renewcommand{\subparagraph}[1]{\oldsubparagraph{#1}\mbox{}}
\fi

% set default figure placement to htbp
\makeatletter
\def\fps@figure{htbp}
\makeatother

\usepackage{booktabs}
\usepackage{longtable}

\usepackage{framed,color}
\definecolor{shadecolor}{RGB}{248,248,248}

\renewcommand{\textfraction}{0.05}
\renewcommand{\topfraction}{0.8}
\renewcommand{\bottomfraction}{0.8}
\renewcommand{\floatpagefraction}{0.75}

\let\oldhref\href
\renewcommand{\href}[2]{#2\footnote{\url{#1}}}

\makeatletter
\newenvironment{kframe}{%
\medskip{}
\setlength{\fboxsep}{.8em}
 \def\at@end@of@kframe{}%
 \ifinner\ifhmode%
  \def\at@end@of@kframe{\end{minipage}}%
  \begin{minipage}{\columnwidth}%
 \fi\fi%
 \def\FrameCommand##1{\hskip\@totalleftmargin \hskip-\fboxsep
 \colorbox{shadecolor}{##1}\hskip-\fboxsep
     % There is no \\@totalrightmargin, so:
     \hskip-\linewidth \hskip-\@totalleftmargin \hskip\columnwidth}%
 \MakeFramed {\advance\hsize-\width
   \@totalleftmargin\z@ \linewidth\hsize
   \@setminipage}}%
 {\par\unskip\endMakeFramed%
 \at@end@of@kframe}
\makeatother

\makeatletter
\@ifundefined{Shaded}{
}{\renewenvironment{Shaded}{\begin{kframe}}{\end{kframe}}}
\@ifpackageloaded{fancyvrb}{%
  % https://github.com/CTeX-org/ctex-kit/issues/331
  \RecustomVerbatimEnvironment{Highlighting}{Verbatim}{commandchars=\\\{\},formatcom=\xeCJKVerbAddon}%
}{}
\makeatother

\usepackage{makeidx}
\makeindex

\urlstyle{tt}

\usepackage{amsthm}
\makeatletter
\def\thm@space@setup{%
  \thm@preskip=8pt plus 2pt minus 4pt
  \thm@postskip=\thm@preskip
}
\makeatother

\frontmatter

\title{老少咸宜Rcpp}
\author{Masaki E. Tsuda 著 jywang 译}
\date{2019-08-19}

\let\BeginKnitrBlock\begin \let\EndKnitrBlock\end
\begin{document}
\maketitle


\thispagestyle{empty}

\setlength{\abovedisplayskip}{-5pt}
\setlength{\abovedisplayshortskip}{-5pt}

{
\setcounter{tocdepth}{2}
\tableofcontents
}
\listoftables
\listoffigures
\chapter*{欢迎}


Rcpp能够让你在R中使用C++。即便你对C++没有很深刻的了解,也可以轻松用R的风格来写C++。此外,Rcpp在易用的同时并不会牺牲执行速度,任何人都可以以此获得高性能的结果。

本文档旨在给那些不熟悉C++的用户提供(使用Rcpp)必要的信息。因此,在某些情况,作者会从Rcpp的角度来解释其用法,让读者容易理解,而不是从C++的视角来力求描述准确。

如果你能为本文档来提供反馈,我会十分感激。

\href{https://github.com/teuder/rcpp4everyone_en}{项目原地址}

\chapter*{本书编译环境}


按照惯例,感谢yihui大佬,以及他的 \textbf{knitr}\index{knitr}
\citep{xie2015} 和 \textbf{bookdown}\index{bookdown}
\citep{R-bookdown}。以下为本书的 R 进程信息:

\begin{Shaded}
\begin{Highlighting}[]
\KeywordTok{sessionInfo}\NormalTok{()}
\end{Highlighting}
\end{Shaded}

\begin{verbatim}
## R version 3.6.0 (2019-04-26)
## Platform: x86_64-w64-mingw32/x64 (64-bit)
## Running under: Windows 10 x64 (build 10586)
## 
## Matrix products: default
## 
## locale:
## [1] LC_COLLATE=Chinese (Simplified)_China.936 
## [2] LC_CTYPE=Chinese (Simplified)_China.936   
## [3] LC_MONETARY=Chinese (Simplified)_China.936
## [4] LC_NUMERIC=C                              
## [5] LC_TIME=Chinese (Simplified)_China.936    
## 
## attached base packages:
## [1] stats     graphics  grDevices utils     datasets 
## [6] methods   base     
## 
## loaded via a namespace (and not attached):
##  [1] compiler_3.6.0  magrittr_1.5    bookdown_0.12  
##  [4] tools_3.6.0     htmltools_0.3.6 yaml_2.2.0     
##  [7] Rcpp_1.0.1      stringi_1.4.3   rmarkdown_1.14 
## [10] knitr_1.24      stringr_1.4.0   xfun_0.8       
## [13] digest_0.6.20   evaluate_0.14
\end{verbatim}

\chapter*{翻译的初衷}


由于研究需要,我在尝试使用C++来实现一些数值算法,求解优化问题。在倒腾Rmarkdown主题美化的时候,关注了\href{https://github.com/yixuan/prettydoc}{prettydoc}。巧的是,在作者yihuan(后面发现也是给\texttt{Rcpp:R与C++的无缝整合}写序的大佬)的\href{https://github.com/yixuan}{github}上,看到了很多相关的工作,比如优化和数值积分,\texttt{Readme}上我看到的都是\texttt{fast}和\texttt{c++}俩词。因此,开始学习\href{RcppCore/Rcpp}{Rcpp}。

\texttt{Rcpp:R与C++的无缝整合}是我见到的第一份完备的Rcpp资料。拿到中文书籍的那一刻,爱不释手,然而读了一遍,大半内容我都是云里雾里。因此只能搁置。于是,算法只能用R写,速度慢也只能认了。后来在一些嵌入式系统上写了一段时间c,回头再看书,内容渐渐明了起来。我这才意识到,此前是我没有达到看书的门槛。

Hadley Wickham在\href{https://adv-r.hadley.nz/}{Advanced
R}中的前言谈到,很多R用户并不是程序员,且R用户追求的是解决问题,而不在意该过程。相信同我一样,很多R用户,其他语言背景(尤其是C/C++)很薄弱。这也意味着,以C++的角度来学Rcpp,会阻挡一部分的R用户迈入Rcpp。

\href{rcpp4everyone_en}{Rcpp for
everyone}一书对这个问题给出了自己的答案。\textbf{This document focuses
on providing necessary information to users who are not familiar with
C++. Therefore, in some cases, I explain usage of Rcpp conceptually
rather than describing accurately from the viewpoint of C++, so that I
hope readers can easily understand it.}
与其设想的描述方式一致,本书的内容偏重于从Rcpp的角度来讲问题,较少涉及到C++的知识。这在保证用户理解的情况下,又能让用户以一种R语言的风格,来写Rcpp代码。在我看来,这可能是对C/C++了解不多的R用户,最简明的Rcpp入门教材。

最后,\texttt{Rcpp\ for\ everyone\ ==\textgreater{}\textgreater{}\ Advanced\ R\ Rcpp部分\ ==\textgreater{}\textgreater{}\ Rcpp:R与C++的无缝整合}应该是我目前能发现的最平滑的Rcpp学习路线。希望本书的中文翻译能对更近一步地降低Rcpp的学习门槛,对大家的学习科研有所帮助。

\BeginKnitrBlock{flushright}
jywang\\
二零一九年八月
\EndKnitrBlock{flushright}

\chapter*{作者简介}\label{author}


Masaki
E.Tsuda,\href{https://github.com/teuder}{github},著有\href{https://teuder.github.io/rcpp4everyone_en/}{Rcpp
for everyone}。

\mainmatter

\chapter{适用Rcpp的情况}\label{situations}

R
在做某些操作的时候较为乏力。如果你需要做下面列出来的一些运算/操作,是时候考虑使用Rcpp了。

\begin{itemize}
\tightlist
\item
  循环,下一次循环依赖此前的循环结果(\texttt{猜测应该是无法直接用向量化来加速循环,所以需要Rcpp})
\item
  遍历一个向量或者矩阵中的每一个元素
\item
  有循环的递归函数
\item
  向量大小动态变化
\item
  需要更高端的数据结构和算法的操作
\end{itemize}

\chapter{安装}\label{install}

在使用Rcpp开发之前,你需要安装一个c++的编译器。

\section{安装C++编译器}\label{cppcompiler}

\subsection{Windows}\label{cppcwindows}

安装\texttt{Rtools}。

参考Rstan的教程可能会有帮助(为Windows安装Rtools)。

\subsection{Mac}\label{cppmac}

安装Xcode命令行工具。在终端执行 \texttt{xcode-select\ -\/-install}命令。

\subsection{Linux}\label{cpplinux}

安装gcc和其他相关的包。 在Ubuntu
Linux中,终端执行\texttt{sudo\ apt-get\ install\ r-base-dev}命令。

\subsection{使用其他编译器}

如果你安装有与上述不同的其他编译器(g++,clang++),在用户的根目录下创建如下的文件。然后在这个文件中设置环境变量。

\textbf{Linux,Mac}

\begin{itemize}
\tightlist
\item
  .R/Makevars
\end{itemize}

\textbf{Windows}

\begin{itemize}
\tightlist
\item
  .R/Makevars.win
\end{itemize}

\textbf{环境变量设定范例}

\begin{verbatim}
CC=/opt/local/bin/gcc-mp-4.7
CXX=/opt/local/bin/g++-mp-4.7
CPLUS_INCLUDE_PATH=/opt/local/include:$CPLUS_INCLUDE_PATH
LD_LIBRARY_PATH=/opt/local/lib:$LD_LIBRARY_PATH
CXXFLAGS= -g0 -O2 -Wall
MAKE=make -j4
\end{verbatim}

\section{安装Rcpp}\label{Rcppinstallation}

用户能通过执行下面代码安装Rcpp。

\begin{Shaded}
\begin{Highlighting}[]
\KeywordTok{install.packages}\NormalTok{(}\StringTok{"Rcpp"}\NormalTok{)}
\end{Highlighting}
\end{Shaded}

\chapter{基本用法}\label{basicUsage}

按照如下三步,即可使用你的Rcpp函数。

\begin{enumerate}
\def\labelenumi{\arabic{enumi}.}
\tightlist
\item
  写(你的函数的)Rcpp代码
\item
  编译上述代码
\item
  执行函数
\end{enumerate}

\section{写Rcpp代码}\label{WriteRcppCode}

下面的代码定义了一个名为\texttt{rcpp\_sum()}的函数,来计算一个向量的元素之和。保存代码块内容至
\texttt{sum.cpp}文件。

\textbf{sum.cpp}

\begin{Shaded}
\begin{Highlighting}[]
\CommentTok{//sum.cpp}
\PreprocessorTok{#include }\ImportTok{<Rcpp.h>}
\KeywordTok{using} \KeywordTok{namespace}\NormalTok{ Rcpp;}
\CommentTok{// [[Rcpp::export]]}
\DataTypeTok{double}\NormalTok{ rcpp_sum(NumericVector v)\{}
    \DataTypeTok{double}\NormalTok{ sum = }\DecValTok{0}\NormalTok{;}
    \ControlFlowTok{for}\NormalTok{(}\DataTypeTok{int}\NormalTok{ i=}\DecValTok{0}\NormalTok{; i<v.length(); ++i)\{}
\NormalTok{        sum += v[i];}
\NormalTok{    \}}
    \ControlFlowTok{return}\NormalTok{(sum);}
\NormalTok{\}}
\end{Highlighting}
\end{Shaded}

\subsection{在Rcpp中定义一个函数的格式}\label{formatForFunction}

下面的代码展示了如何定义一个Rcpp函数的基本格式。

\begin{Shaded}
\begin{Highlighting}[]
\PreprocessorTok{#include}\ImportTok{<Rcpp.h>}
\KeywordTok{using} \KeywordTok{namespace}\NormalTok{ Rcpp;}
\CommentTok{// [[Rcpp::export]]}
\NormalTok{RETURN_TYPE FUNCTION_NAME(ARGUMENT_TYPE ARGUMENT)\{}
    \CommentTok{//do something}
    \ControlFlowTok{return}\NormalTok{ RETURN_VALUE;}
\NormalTok{\}}
\end{Highlighting}
\end{Shaded}

\begin{itemize}
\item
  \texttt{\#include\textless{}Rcpp.h\textgreater{}} :
  保证你能使用Rcpp包中定义的的类和函数
\item
  \texttt{//\ {[}{[}Rcpp::export{]}{]}}:在这行代码下定义的函数,才能(在后面的步骤中)被R获取.
\item
  \texttt{using\ namespace\ Rcpp;} :
  这行代码是可选的。如果你不写,那么你需要在特定的类和函数前面加上前缀\texttt{Rcpp::}.
  (比如, \texttt{Rcpp::NumericVector})
\item
  \texttt{RETURN\_TYPE\ FUNCTION\_NAME(ARGMENT\_TYPE\ ARGMENT)\{\}}:需要指定函数的返回值和参数的类型,以及函数名.
\item
  \texttt{return\ RETURN\_VALUE;}:\texttt{return}声明强制返回一个值,但是如果你不返回值(例如,\texttt{RETURN\_TYPE}是\texttt{void}型,即空),那么可以忽略\texttt{return}。
\end{itemize}

\section{编译代码}\label{compile}

函数\texttt{Rcpp::sourceCpp()}会编译上述源代码并在R中加载。

\begin{Shaded}
\begin{Highlighting}[]
\KeywordTok{library}\NormalTok{(Rcpp)}
\KeywordTok{sourceCpp}\NormalTok{(}\StringTok{'sum.cpp'}\NormalTok{)}
\end{Highlighting}
\end{Shaded}

\section{执行函数}\label{Execute}

你可以像使用其他R函数一样,来使用加载好的Rcpp函数。

\begin{Shaded}
\begin{Highlighting}[]
\OperatorTok{>}\StringTok{ }\KeywordTok{rcpp_sum}\NormalTok{(}\DecValTok{1}\OperatorTok{:}\DecValTok{10}\NormalTok{)}
\NormalTok{[}\DecValTok{1}\NormalTok{] }\DecValTok{55}
\OperatorTok{>}\StringTok{ }\KeywordTok{sum}\NormalTok{(}\DecValTok{1}\OperatorTok{:}\DecValTok{10}\NormalTok{)}
\NormalTok{[}\DecValTok{1}\NormalTok{] }\DecValTok{55}
\end{Highlighting}
\end{Shaded}

\chapter{将Rcpp嵌入R代码}\label{embedd}

三种方式可以让你在R代码中写Rcpp代码,即使用\texttt{sourceCpp()},
\texttt{cppFunction()}, \texttt{evalCpp()}函数.

\section{sourceCpp()}\label{sourcecpp}

不同于\ref{compile}中需要加载外部的Rcpp文件,你可以直接在R中,写Rcpp代码,并且将其保存为一个字符串对象,然后利用\texttt{sourceCpp()}加载这个对象。

\begin{Shaded}
\begin{Highlighting}[]
\NormalTok{src <-}
\StringTok{"#include <Rcpp.h>}
\StringTok{using namespace Rcpp;}
\StringTok{// [[Rcpp::export]]}
\StringTok{double rcpp_sum(NumericVector v)\{}
\StringTok{  double sum = 0;}
\StringTok{  for(int i=0; i<v.length(); ++i)\{}
\StringTok{    sum += v[i];}
\StringTok{  \}}
\StringTok{  return(sum);}
\StringTok{\}"}
\KeywordTok{sourceCpp}\NormalTok{(}\DataTypeTok{code =}\NormalTok{ src)}
\KeywordTok{rcpp_sum}\NormalTok{(}\DecValTok{1}\OperatorTok{:}\DecValTok{10}\NormalTok{)}
\end{Highlighting}
\end{Shaded}

\section{cppFunction()}\label{cppFunction}

\texttt{cppFunction()}提供了一种更加便利的方式来构建\textbf{单一的}Rcpp函数(采用sourceCpp的方式可以写多个Rcpp函数被R调用)。使用\texttt{cppFunction()}时,可以忽略\texttt{\#include\ \textless{}Rcpp.h\textgreater{}}
和 \texttt{using\ namespase\ Rcpp;}。

\begin{Shaded}
\begin{Highlighting}[]
\NormalTok{src <-}
\StringTok{  "double rcpp_sum(NumericVector v)\{}
\StringTok{    double sum = 0;}
\StringTok{    for(int i=0; i<v.length(); ++i)\{}
\StringTok{      sum += v[i];}
\StringTok{    \}}
\StringTok{    return(sum);}
\StringTok{  \}}
\StringTok{  "}
\NormalTok{Rcpp}\OperatorTok{::}\KeywordTok{cppFunction}\NormalTok{(src)}
\KeywordTok{rcpp_sum}\NormalTok{(}\DecValTok{1}\OperatorTok{:}\DecValTok{10}\NormalTok{)}
\end{Highlighting}
\end{Shaded}

\section{evalCpp()}\label{evalcpp}

可以使用\texttt{evalCpp()}直接执行单个的C++声明。

\begin{Shaded}
\begin{Highlighting}[]
\CommentTok{# Showing maximum value of double.}
\KeywordTok{evalCpp}\NormalTok{(}\StringTok{'std::numeric_limits<double>::max()'}\NormalTok{)}
\end{Highlighting}
\end{Shaded}

\chapter{C++11}\label{cpp11}

C++11是在2011年新建立的C++标准,该标准引入了新的函数(functionalities)和符号(notations)。相较于之前的标准,很多新的特性使得C++对于初学者而言更为简单。本文档会对C++11的新特性做很多探索。

\textbf{重要:代码示例是在默认C++11可得的情况下写的。}

\section{开启C++11}\label{enable}

为了开启C++11,在你的Rcpp代码中任意一处加入下面的代码。

\begin{Shaded}
\begin{Highlighting}[]
\CommentTok{// [[Rcpp::plugins("cpp11")]]}
\end{Highlighting}
\end{Shaded}

\section{推荐的 C++11 特性}\label{recommend}

\subsection{初始化}\label{initializer}

使用 \texttt{\{\}}来初始化变量.

\begin{Shaded}
\begin{Highlighting}[]
\CommentTok{// 初始化向量}
\CommentTok{// 下面三行代码等同于R中的 c (1, 2, 3).}
\NormalTok{NumericVector v1 = NumericVector::create(}\FloatTok{1.0}\NormalTok{, }\FloatTok{2.0}\NormalTok{, }\FloatTok{3.0}\NormalTok{);}
\NormalTok{NumericVector v2 = \{}\FloatTok{1.0}\NormalTok{, }\FloatTok{2.0}\NormalTok{, }\FloatTok{3.0}\NormalTok{\};}
\NormalTok{NumericVector v3   \{}\FloatTok{1.0}\NormalTok{, }\FloatTok{2.0}\NormalTok{, }\FloatTok{3.0}\NormalTok{\}; }\CommentTok{// 你可以忽略 "=".}
\end{Highlighting}
\end{Shaded}

\subsection{auto}\label{auto}

使用\texttt{auto}关键字,根据赋值,来推断变量的类型

\begin{Shaded}
\begin{Highlighting}[]
\CommentTok{// 变量 "i" 会是 int 型}
\KeywordTok{auto}\NormalTok{  i  = }\DecValTok{4}\NormalTok{;}
\NormalTok{NumericVector v;}
\CommentTok{// 变量 "it" 会是 NumericVector::iterator }
\KeywordTok{auto}\NormalTok{ it = v.begin();}
\end{Highlighting}
\end{Shaded}

\subsection{decltype}\label{decltype}

通过使用 \texttt{decltype},你能声明变量的类型与已存在的变量相同.

\begin{Shaded}
\begin{Highlighting}[]
\DataTypeTok{int}\NormalTok{ i;}
\KeywordTok{decltype}\NormalTok{(i) x; }\CommentTok{// 变量 "x" 会是 int 型}
\end{Highlighting}
\end{Shaded}

\subsection{基于范围的for循环}\label{rangefor}

能用一种比较R风格的方式来写for循环。

\begin{Shaded}
\begin{Highlighting}[]
\NormalTok{IntegerVector v\{}\DecValTok{1}\NormalTok{,}\DecValTok{2}\NormalTok{,}\DecValTok{3}\NormalTok{\};}
\DataTypeTok{int}\NormalTok{ sum=}\DecValTok{0}\NormalTok{;}
\ControlFlowTok{for}\NormalTok{(}\KeywordTok{auto}\NormalTok{& x : v) \{}
\NormalTok{  sum += x;}
\NormalTok{\}}
\end{Highlighting}
\end{Shaded}

\subsection{Lambda 表达式}\label{Lambda}

你能使用lambda表达式来创建一个函数对象。函数对象常用于作为未命名函数传递给其他函数。

Lambda表达式的形式为 \texttt{{[}{]}()\{\}}.

在 \texttt{{[}{]}}中,写你希望在函数对象中使用的局部变量的列表.

\begin{itemize}
\tightlist
\item
  \texttt{{[}{]}} 不允许函数对象获取所有的局部变量.
\item
  \texttt{{[}={]}} 传值,将所有局部变量的值传递给函数对象.
\item
  \texttt{{[}\&{]}} 引用,函数直接引用所有的局部变量的值.
\item
  \texttt{{[}=x,\ \&y{]}} ``x'' 传值给函数对象,而``y''
  可以被函数对象直接引用.
\end{itemize}

在 \texttt{()}中,写传递给函数的参数列表.

在\texttt{\{\}}中,写想做的操作.

\textbf{lambda 表达式的返回类型}

函数对象的返回类型会被自动地设定为\texttt{\{\}}操作中返回值的类型。如果你希望显式定义返回类型,可以按照\texttt{{[}{]}()-\textgreater{}int\{\}}的方式。

\textbf{例子}
下面的例子展示了如何使用lambda表达式。可以按照R的风格来写某些类型的C++代码。

\emph{R 例子}

\begin{Shaded}
\begin{Highlighting}[]
\NormalTok{v <-}\StringTok{ }\KeywordTok{c}\NormalTok{(}\DecValTok{1}\NormalTok{,}\DecValTok{2}\NormalTok{,}\DecValTok{3}\NormalTok{,}\DecValTok{4}\NormalTok{,}\DecValTok{5}\NormalTok{)}
\NormalTok{A <-}\StringTok{ }\FloatTok{2.0}
\NormalTok{res <-}
\StringTok{  }\KeywordTok{sapply}\NormalTok{(v, }\ControlFlowTok{function}\NormalTok{(x)\{A}\OperatorTok{*}\NormalTok{x\})}
\end{Highlighting}
\end{Shaded}

\begin{quote}
译者:上面的sapply函数,将function(x)\{A*x\}作用于v中的每个元素。这是很典型的向量化编程,可以看apply函数族来对此有更深的了解。
\end{quote}

\emph{Rcpp 例子}

\begin{Shaded}
\begin{Highlighting}[]
\CommentTok{// [[Rcpp::plugins("cpp11")]] }
\CommentTok{// [[Rcpp::export]]}
\NormalTok{NumericVector rcpp_lambda_1()\{}
\NormalTok{  NumericVector v = \{}\DecValTok{1}\NormalTok{,}\DecValTok{2}\NormalTok{,}\DecValTok{3}\NormalTok{,}\DecValTok{4}\NormalTok{,}\DecValTok{5}\NormalTok{\};}
  \DataTypeTok{double}\NormalTok{ A = }\FloatTok{2.0}\NormalTok{;}
\NormalTok{  NumericVector res =}
\NormalTok{    sapply(v, [&](}\DataTypeTok{double}\NormalTok{ x)\{}\ControlFlowTok{return}\NormalTok{ A*x;\});}
  \ControlFlowTok{return}\NormalTok{ res;}
\NormalTok{\}}
\end{Highlighting}
\end{Shaded}

\begin{quote}
译者:上面代码中也使用sapply函数,将R中的function(x)(A*x)
用lambda表达式的方式替代。而sapply,则是吃到了Rcpp给我们提供的语法糖(sugar)。如果要对语法糖有更深的了解,可以阅读Rcpp:R与C++的无缝整合中的相关章节。
\end{quote}

\chapter{信息打印}\label{print}

你可以通过\texttt{Rprintf()}和\texttt{Rcout}在R的控制台上来打印对象的信息和值。

\texttt{REprintf()} 和 \texttt{Rcerr} 可用于打印报错信息。

\section{Rcout, Rcerr}\label{Rcout}

使用\texttt{Rcout} 和 \texttt{Rcerr}的方式与
使用\texttt{std::cout}和\texttt{std::cerr}方式相同。将你想要输出的信息,按照特定的顺序,使用\texttt{\textless{}\textless{}}输出符号链接。\textbf{当在}\texttt{\textless{}\textless{}}\textbf{前是向量对象时},程序会打印向量的所有元素。

\begin{Shaded}
\begin{Highlighting}[]
\CommentTok{// [[Rcpp::export]]}
\DataTypeTok{void}\NormalTok{ rcpp_rcout(NumericVector v)\{}
  \CommentTok{// 打印向量的所有元素值}
\NormalTok{  Rcout << }\StringTok{"The value of v : "}\NormalTok{ << v << }\StringTok{"}\SpecialCharTok{\textbackslash{}n}\StringTok{"}\NormalTok{;}
  \CommentTok{// 打印报错信息}
\NormalTok{  Rcerr << }\StringTok{"Error message}\SpecialCharTok{\textbackslash{}n}\StringTok{"}\NormalTok{;}
\NormalTok{\}}
\end{Highlighting}
\end{Shaded}

\section{Rprintf(), REprintf()}\label{Rprintf}

使用\texttt{Rprintf()} 和 \texttt{REprintf()} 的方式和
\texttt{std::printf()}相同, 函数会按照指定的格式打印信息。

\begin{Shaded}
\begin{Highlighting}[]
\NormalTok{Rprintf( format, variables)}
\end{Highlighting}
\end{Shaded}

在\texttt{format}字符串中,你可以使用下面的格式指示符来打印变了的值。如果要打印多个变量,你需要按照对应的格式字符串的顺序,来排列你的变量。

下面只列举了一部分的格式指示符,可以参考其他的文档以深入探究(如,\href{http://www.cplusplus.com/reference/cstdio/printf/}{cplusplus.com})。

\begin{longtable}[]{@{}cl@{}}
\toprule
specifier & explanation\tabularnewline
\midrule
\endhead
\texttt{\%i} & 打印 signed integer (\texttt{int})\tabularnewline
\texttt{\%u} & 打印 unsigned integer
(\texttt{unsigned\ int})\tabularnewline
\texttt{\%f} & 打印 floating point number
(\texttt{double})\tabularnewline
\texttt{\%e} & 打印 floating point number (\texttt{double}) in
exponential style\tabularnewline
\texttt{\%s} & 打印 C string (\texttt{char*})\tabularnewline
\bottomrule
\end{longtable}

此外,\texttt{Rprintf()} 和 \texttt{REprintf()}
\textbf{只能打印在标准的C语言中已有的数据类型},因此,用户并不能直接传递Rcpp包中定义的数据类型,如\texttt{NumericVector}给\texttt{Rprintf()}打印。如果你想进行此类操作,那么你可能需要逐个元素进行传递打印,代码如下。

\begin{Shaded}
\begin{Highlighting}[]
\CommentTok{// [[Rcpp::export]]}
\DataTypeTok{void}\NormalTok{ rcpp_rprintf(NumericVector v)\{}
    \CommentTok{// printing values of all the elements of Rcpp vector  }
    \ControlFlowTok{for}\NormalTok{(}\DataTypeTok{int}\NormalTok{ i=}\DecValTok{0}\NormalTok{; i<v.length(); ++i)\{}
\NormalTok{        Rprintf(}\StringTok{"the value of v[}\SpecialCharTok{%i}\StringTok{] : }\SpecialCharTok{%f}\StringTok{ }\SpecialCharTok{\textbackslash{}n}\StringTok{"}\NormalTok{, i, v[i]);}
\NormalTok{    \}}
\NormalTok{\}}
\end{Highlighting}
\end{Shaded}

\chapter{数据类型}\label{datatypes}

Rcpp提供了R中所有的基本数据类型。通过使用这些数据类型,你能够直接使用在R中的对象。

\section{向量和矩阵}\label{vectorMatrix}

下面7种数据类型在R中被经常使用。

\texttt{logical} \texttt{integer} \texttt{numeric} \texttt{complex}
\texttt{character} \texttt{Date} \texttt{POSIXct}

上面的7种类型与Rcpp中的向量(\texttt{vector})类型和矩阵(\texttt{matrix})类型是对应的(比如,有
logicalVector,integerVector等类型,上面7种基本的数据类型都可以在后面加上Vector或者Matrix)。

本文档中, \texttt{Vector} 和 \texttt{Matrix}
用于特指Rcpp中所有的向量和矩阵类型。

下表中展示了R/Rcpp/C++中对应的数据类型。

\begin{longtable}[]{@{}cccccc@{}}
\toprule
\begin{minipage}[b]{0.07\columnwidth}\centering\strut
Value\strut
\end{minipage} & \begin{minipage}[b]{0.07\columnwidth}\centering\strut
R vector\strut
\end{minipage} & \begin{minipage}[b]{0.07\columnwidth}\centering\strut
Rcpp vector\strut
\end{minipage} & \begin{minipage}[b]{0.07\columnwidth}\centering\strut
Rcpp matrix\strut
\end{minipage} & \begin{minipage}[b]{0.07\columnwidth}\centering\strut
Rcpp scalar\strut
\end{minipage} & \begin{minipage}[b]{0.07\columnwidth}\centering\strut
C++ scalar\strut
\end{minipage}\tabularnewline
\midrule
\endhead
\begin{minipage}[t]{0.07\columnwidth}\centering\strut
Logical\strut
\end{minipage} & \begin{minipage}[t]{0.07\columnwidth}\centering\strut
\texttt{logical}\strut
\end{minipage} & \begin{minipage}[t]{0.07\columnwidth}\centering\strut
\texttt{LogicalVector}\strut
\end{minipage} & \begin{minipage}[t]{0.07\columnwidth}\centering\strut
\texttt{LogicalMatrix}\strut
\end{minipage} & \begin{minipage}[t]{0.07\columnwidth}\centering\strut
-\strut
\end{minipage} & \begin{minipage}[t]{0.07\columnwidth}\centering\strut
\texttt{bool}\strut
\end{minipage}\tabularnewline
\begin{minipage}[t]{0.07\columnwidth}\centering\strut
Integer\strut
\end{minipage} & \begin{minipage}[t]{0.07\columnwidth}\centering\strut
\texttt{integer}\strut
\end{minipage} & \begin{minipage}[t]{0.07\columnwidth}\centering\strut
\texttt{IntegerVector}\strut
\end{minipage} & \begin{minipage}[t]{0.07\columnwidth}\centering\strut
\texttt{IntegerMatrix}\strut
\end{minipage} & \begin{minipage}[t]{0.07\columnwidth}\centering\strut
-\strut
\end{minipage} & \begin{minipage}[t]{0.07\columnwidth}\centering\strut
\texttt{int}\strut
\end{minipage}\tabularnewline
\begin{minipage}[t]{0.07\columnwidth}\centering\strut
Real\strut
\end{minipage} & \begin{minipage}[t]{0.07\columnwidth}\centering\strut
\texttt{numeric}\strut
\end{minipage} & \begin{minipage}[t]{0.07\columnwidth}\centering\strut
\texttt{NumericVector}\strut
\end{minipage} & \begin{minipage}[t]{0.07\columnwidth}\centering\strut
\texttt{NumericMatrix}\strut
\end{minipage} & \begin{minipage}[t]{0.07\columnwidth}\centering\strut
-\strut
\end{minipage} & \begin{minipage}[t]{0.07\columnwidth}\centering\strut
\texttt{double}\strut
\end{minipage}\tabularnewline
\begin{minipage}[t]{0.07\columnwidth}\centering\strut
Complex\strut
\end{minipage} & \begin{minipage}[t]{0.07\columnwidth}\centering\strut
\texttt{complex}\strut
\end{minipage} & \begin{minipage}[t]{0.07\columnwidth}\centering\strut
\texttt{ComplexVector}\strut
\end{minipage} & \begin{minipage}[t]{0.07\columnwidth}\centering\strut
\texttt{ComplexMatrix}\strut
\end{minipage} & \begin{minipage}[t]{0.07\columnwidth}\centering\strut
\texttt{Rcomplex}\strut
\end{minipage} & \begin{minipage}[t]{0.07\columnwidth}\centering\strut
\texttt{complex}\strut
\end{minipage}\tabularnewline
\begin{minipage}[t]{0.07\columnwidth}\centering\strut
String\strut
\end{minipage} & \begin{minipage}[t]{0.07\columnwidth}\centering\strut
\texttt{character}\strut
\end{minipage} & \begin{minipage}[t]{0.07\columnwidth}\centering\strut
\texttt{CharacterVector} (\texttt{StringVector})\strut
\end{minipage} & \begin{minipage}[t]{0.07\columnwidth}\centering\strut
\texttt{CharacterMatrix} (\texttt{StringMatrix})\strut
\end{minipage} & \begin{minipage}[t]{0.07\columnwidth}\centering\strut
\texttt{String}\strut
\end{minipage} & \begin{minipage}[t]{0.07\columnwidth}\centering\strut
\texttt{string}\strut
\end{minipage}\tabularnewline
\begin{minipage}[t]{0.07\columnwidth}\centering\strut
Date\strut
\end{minipage} & \begin{minipage}[t]{0.07\columnwidth}\centering\strut
\texttt{Date}\strut
\end{minipage} & \begin{minipage}[t]{0.07\columnwidth}\centering\strut
\texttt{DateVector}\strut
\end{minipage} & \begin{minipage}[t]{0.07\columnwidth}\centering\strut
-\strut
\end{minipage} & \begin{minipage}[t]{0.07\columnwidth}\centering\strut
\texttt{Date}\strut
\end{minipage} & \begin{minipage}[t]{0.07\columnwidth}\centering\strut
-\strut
\end{minipage}\tabularnewline
\begin{minipage}[t]{0.07\columnwidth}\centering\strut
Datetime\strut
\end{minipage} & \begin{minipage}[t]{0.07\columnwidth}\centering\strut
\texttt{POSIXct}\strut
\end{minipage} & \begin{minipage}[t]{0.07\columnwidth}\centering\strut
\texttt{DatetimeVector}\strut
\end{minipage} & \begin{minipage}[t]{0.07\columnwidth}\centering\strut
-\strut
\end{minipage} & \begin{minipage}[t]{0.07\columnwidth}\centering\strut
\texttt{Datetime}\strut
\end{minipage} & \begin{minipage}[t]{0.07\columnwidth}\centering\strut
\texttt{time\_t}\strut
\end{minipage}\tabularnewline
\bottomrule
\end{longtable}

\section{data.frame, list, S3, S4}\label{dataframe}

除了向量和矩阵,在R中海油一些数据结构,比如
data.frame,list,S3和S4类。所有这些数据结构同样也可以在Rcpp中处理。

\begin{longtable}[]{@{}cc@{}}
\toprule
R & Rcpp\tabularnewline
\midrule
\endhead
\texttt{data.frame} & \texttt{DataFrame}\tabularnewline
\texttt{list} & \texttt{List}\tabularnewline
S3 class & \texttt{List}\tabularnewline
S4 class & \texttt{S4}\tabularnewline
\bottomrule
\end{longtable}

在Rcpp中,\texttt{Vector}, \texttt{DataFrame},
\texttt{List}都以向量的方式实现。即,\texttt{Vector}
是一个元素全部为标量的向量,\texttt{DataFrame}是元素全部为向量的向量,\texttt{List}是元素为各种各样数据类型的向量。因此,在Rcpp中\texttt{Vector},
\texttt{DataFrame}, \texttt{List} 有很多共同的成员函数。

\chapter{Vector类}\label{vector}

\section{创建向量对象}\label{create-vector-object}

你可以使用下面的几种方法来创建向量对象。

\begin{Shaded}
\begin{Highlighting}[]
\CommentTok{// 等价于 v <- rep(0, 3)}
\NormalTok{NumericVector v (}\DecValTok{3}\NormalTok{);}

\CommentTok{// 等价于 v <- rep(1, 3)}
\NormalTok{NumericVector v (}\DecValTok{3}\NormalTok{,}\DecValTok{1}\NormalTok{);}

\CommentTok{// 等价于 v <- c(1,2,3) }
\CommentTok{// C++11 初始化列表}
\NormalTok{NumericVector v = \{}\DecValTok{1}\NormalTok{,}\DecValTok{2}\NormalTok{,}\DecValTok{3}\NormalTok{\}; }

\CommentTok{// 等价于 v <- c(1,2,3)}
\NormalTok{NumericVector v = NumericVector::create(}\DecValTok{1}\NormalTok{,}\DecValTok{2}\NormalTok{,}\DecValTok{3}\NormalTok{);}

\CommentTok{// 命名向量 等价于 v <- c(x=1, y=2, z=3)}
\NormalTok{NumericVector v =}
\NormalTok{  NumericVector::create(Named(}\StringTok{"x"}\NormalTok{,}\DecValTok{1}\NormalTok{), Named(}\StringTok{"y"}\NormalTok{)=}\DecValTok{2}\NormalTok{ , _[}\StringTok{"z"}\NormalTok{]=}\DecValTok{3}\NormalTok{);}
\end{Highlighting}
\end{Shaded}

\section{获取向量元素}\label{Accessing-vector-elements}

你可以使用\texttt{{[}{]}} 或
\texttt{()}运算符来获取一个向量的个别元素。两种操作符都接受
数值向量/整型向量(NumericVector/IntegerVector)
的数值索引,字符向量的元素名索引和逻辑向量。\texttt{{[}{]}}运算符会忽略边界溢出,而\texttt{()}
运算符会抛出\texttt{index\_out\_of\_bounds}错误。

\textbf{需要注意的是 C++中的向量索引开始于0}

\begin{Shaded}
\begin{Highlighting}[]
\CommentTok{// [[Rcpp::export]]}
\DataTypeTok{void}\NormalTok{ rcpp_vector_access()\{}

  \CommentTok{// 创建向量}
\NormalTok{  NumericVector v  \{}\DecValTok{10}\NormalTok{,}\DecValTok{20}\NormalTok{,}\DecValTok{30}\NormalTok{,}\DecValTok{40}\NormalTok{,}\DecValTok{50}\NormalTok{\};}
  \CommentTok{// 设置元素名称}
\NormalTok{  v.names() = CharacterVector(\{}\StringTok{"A"}\NormalTok{,}\StringTok{"B"}\NormalTok{,}\StringTok{"C"}\NormalTok{,}\StringTok{"D"}\NormalTok{,}\StringTok{"E"}\NormalTok{\});}
  
  \CommentTok{// 准备向量索引}
\NormalTok{  NumericVector   numeric = \{}\DecValTok{1}\NormalTok{,}\DecValTok{3}\NormalTok{\};}
\NormalTok{  IntegerVector   integer = \{}\DecValTok{1}\NormalTok{,}\DecValTok{3}\NormalTok{\};}
\NormalTok{  CharacterVector character = \{}\StringTok{"B"}\NormalTok{,}\StringTok{"D"}\NormalTok{\};}
\NormalTok{  LogicalVector   logical = \{}\KeywordTok{false}\NormalTok{, }\KeywordTok{true}\NormalTok{, }\KeywordTok{false}\NormalTok{, }\KeywordTok{true}\NormalTok{, }\KeywordTok{false}\NormalTok{\};}
  
  \CommentTok{// 根据向量索引获取向量元素值}
  \DataTypeTok{double}\NormalTok{ x1 = v[}\DecValTok{0}\NormalTok{];}
  \DataTypeTok{double}\NormalTok{ x2 = v[}\StringTok{"A"}\NormalTok{];}
\NormalTok{  NumericVector res1 = v[numeric];}
\NormalTok{  NumericVector res2 = v[integer];}
\NormalTok{  NumericVector res3 = v[character];}
\NormalTok{  NumericVector res4 = v[logical];}
  
  \CommentTok{// 向量元素赋值}
\NormalTok{  v[}\DecValTok{0}\NormalTok{]   = }\DecValTok{100}\NormalTok{;}
\NormalTok{  v[}\StringTok{"A"}\NormalTok{] = }\DecValTok{100}\NormalTok{;}
\NormalTok{  NumericVector v2 \{}\DecValTok{100}\NormalTok{,}\DecValTok{200}\NormalTok{\};}
\NormalTok{  v[numeric]   = v2;}
\NormalTok{  v[integer]   = v2;}
\NormalTok{  v[character] = v2;}
\NormalTok{  v[logical]   = v2;}
\NormalTok{\}}
\end{Highlighting}
\end{Shaded}

\section{成员函数}\label{Member-functions}

成员函数(也被称作方法)是某个对象中的函数。你可以以\texttt{v.f()}的形式来调用对象\texttt{v}中的成员函数\texttt{f()}。

\begin{Shaded}
\begin{Highlighting}[]
\NormalTok{NumericVector v = \{}\DecValTok{1}\NormalTok{,}\DecValTok{2}\NormalTok{,}\DecValTok{3}\NormalTok{,}\DecValTok{4}\NormalTok{,}\DecValTok{5}\NormalTok{\};}

\CommentTok{// 调用成员函数length(),求对象v的长度}
\DataTypeTok{int}\NormalTok{ n = v.length(); }\CommentTok{// 5}
\end{Highlighting}
\end{Shaded}

Rcpp中,向量对象的成员函数列举如下。

\subsection{length(), size()}\label{vector-length-size}

返回该向量对象中元素的个数。

\begin{Shaded}
\begin{Highlighting}[]
\CommentTok{//test.cpp 文件}
\PreprocessorTok{#include }\ImportTok{<Rcpp.h>}
\KeywordTok{using} \KeywordTok{namespace}\NormalTok{ Rcpp;}

\CommentTok{// [[Rcpp::export]]}
\DataTypeTok{int}\NormalTok{ test(NumericVector v) \{}
  
\NormalTok{  Rcout << v.length() << }\StringTok{'}\SpecialCharTok{\textbackslash{}n}\StringTok{'}\NormalTok{;}
\NormalTok{  Rcout << v.size() << }\StringTok{'}\SpecialCharTok{\textbackslash{}n}\StringTok{'}\NormalTok{;}
  \ControlFlowTok{return} \DecValTok{0}\NormalTok{;}
\NormalTok{\}}
\end{Highlighting}
\end{Shaded}

在R中,运行结果为:

\begin{Shaded}
\begin{Highlighting}[]
\OperatorTok{>}\StringTok{ }\KeywordTok{sourceCpp}\NormalTok{(}\StringTok{"test.cpp"}\NormalTok{)}

\OperatorTok{>}\StringTok{ }\KeywordTok{test}\NormalTok{(}\KeywordTok{c}\NormalTok{(}\DecValTok{1}\OperatorTok{:}\DecValTok{10}\NormalTok{))}
\DecValTok{10}
\DecValTok{10}
\NormalTok{[}\DecValTok{1}\NormalTok{] }\DecValTok{0}
\end{Highlighting}
\end{Shaded}

\subsection{names()}\label{vector-names}

以字符向量的形式,返回该向量的元素名称。

\begin{Shaded}
\begin{Highlighting}[]
\CommentTok{//test.cpp 文件}
\PreprocessorTok{#include }\ImportTok{<Rcpp.h>}
\KeywordTok{using} \KeywordTok{namespace}\NormalTok{ Rcpp;}

\CommentTok{// [[Rcpp::export]]}
\NormalTok{CharacterVector test(NumericVector v) \{}
  
  \ControlFlowTok{return}\NormalTok{ v.names();}
\NormalTok{\}}
\end{Highlighting}
\end{Shaded}

在R中,运行结果为:

\begin{Shaded}
\begin{Highlighting}[]
\OperatorTok{>}\StringTok{ }\KeywordTok{sourceCpp}\NormalTok{(}\StringTok{"test.cpp"}\NormalTok{)}

\OperatorTok{>}\StringTok{ }\KeywordTok{test}\NormalTok{(}\KeywordTok{c}\NormalTok{(}\DataTypeTok{a =} \DecValTok{1}\NormalTok{,}\DataTypeTok{b =} \DecValTok{2}\NormalTok{, }\DataTypeTok{c =} \DecValTok{3}\NormalTok{))}
\NormalTok{[}\DecValTok{1}\NormalTok{] }\StringTok{"a"} \StringTok{"b"} \StringTok{"c"}
\end{Highlighting}
\end{Shaded}

\subsection{offset( name ), findName( name )}\label{offsetname}

按照指定字符串\texttt{name}的方式,返回对应元素的数值索引。

\begin{Shaded}
\begin{Highlighting}[]
\CommentTok{//test.cpp 文件}
\PreprocessorTok{#include }\ImportTok{<Rcpp.h>}
\KeywordTok{using} \KeywordTok{namespace}\NormalTok{ Rcpp;}

\CommentTok{// [[Rcpp::export]]}
\DataTypeTok{int}\NormalTok{ test(NumericVector v, }\BuiltInTok{std::}\NormalTok{string name) \{}
  
  \ControlFlowTok{return}\NormalTok{ v.offset(name);}
\NormalTok{\}}
\end{Highlighting}
\end{Shaded}

在R中,运行结果为:

\begin{Shaded}
\begin{Highlighting}[]
\OperatorTok{>}\StringTok{ }\KeywordTok{sourceCpp}\NormalTok{(}\StringTok{"test.cpp"}\NormalTok{)}

\OperatorTok{>}\StringTok{ }\KeywordTok{test}\NormalTok{(}\KeywordTok{c}\NormalTok{(}\DataTypeTok{a =} \DecValTok{1}\NormalTok{,}\DataTypeTok{b =} \DecValTok{2}\NormalTok{, }\DataTypeTok{c =} \DecValTok{3}\NormalTok{),}\StringTok{"a"}\NormalTok{)}
\NormalTok{[}\DecValTok{1}\NormalTok{] }\DecValTok{0}
\end{Highlighting}
\end{Shaded}

\subsection{offset( i )}\label{offseti}

函数在检查数值索引i没有超过边界后,返回该索引。

举例说明,在\texttt{test.cpp}文件中键入以下代码,

\begin{Shaded}
\begin{Highlighting}[]
\PreprocessorTok{#include }\ImportTok{<Rcpp.h>}
\KeywordTok{using} \KeywordTok{namespace}\NormalTok{ Rcpp;}

\CommentTok{// [[Rcpp::export]]}
\DataTypeTok{int}\NormalTok{ test(NumericVector v, }\DataTypeTok{int}\NormalTok{ i) \{}
  \ControlFlowTok{return}\NormalTok{ v.offset(i);}
\NormalTok{\}}
\end{Highlighting}
\end{Shaded}

在R中,运行结果为:

\begin{Shaded}
\begin{Highlighting}[]
\OperatorTok{>}\StringTok{ }\KeywordTok{sourceCpp}\NormalTok{(}\StringTok{'test.cpp'}\NormalTok{)}

\OperatorTok{>}\StringTok{ }\KeywordTok{test}\NormalTok{(}\KeywordTok{c}\NormalTok{(}\DecValTok{11}\NormalTok{,}\DecValTok{22}\NormalTok{,}\DecValTok{33}\NormalTok{,}\DecValTok{44}\NormalTok{,}\DecValTok{55}\NormalTok{),}\DecValTok{2}\NormalTok{)}
\NormalTok{[}\DecValTok{1}\NormalTok{] }\DecValTok{2}
\end{Highlighting}
\end{Shaded}

\subsection{fill( x )}\label{fillx}

将该向量的所有元素用标量\texttt{x}填充。

举例说明,在\texttt{test.cpp}文件中键入以下代码,

\begin{Shaded}
\begin{Highlighting}[]
\PreprocessorTok{#include }\ImportTok{<Rcpp.h>}
\KeywordTok{using} \KeywordTok{namespace}\NormalTok{ Rcpp;}

\CommentTok{// [[Rcpp::export]]}
\NormalTok{NumericVector test(NumericVector v, }\DataTypeTok{double}\NormalTok{ i) \{}
\NormalTok{  v.fill(i);}
  \ControlFlowTok{return}\NormalTok{ v;}
\NormalTok{\}}
\end{Highlighting}
\end{Shaded}

\begin{Shaded}
\begin{Highlighting}[]
\OperatorTok{>}\StringTok{ }\KeywordTok{sourceCpp}\NormalTok{(}\StringTok{"test.cpp"}\NormalTok{)}

\OperatorTok{>}\StringTok{ }\KeywordTok{test}\NormalTok{(}\KeywordTok{c}\NormalTok{(}\DecValTok{1}\NormalTok{,}\DecValTok{2}\NormalTok{,}\DecValTok{3}\NormalTok{,}\DecValTok{4}\NormalTok{,}\DecValTok{5}\NormalTok{),}\DecValTok{6}\NormalTok{)}
\NormalTok{[}\DecValTok{1}\NormalTok{] }\DecValTok{6} \DecValTok{6} \DecValTok{6} \DecValTok{6} \DecValTok{6}
\end{Highlighting}
\end{Shaded}

\subsection{sort()}\label{sort}

将该向量对象中的元素升序排列。

举例说明,在\texttt{test.cpp}文件中键入以下代码,

\begin{Shaded}
\begin{Highlighting}[]
\PreprocessorTok{#include }\ImportTok{<Rcpp.h>}
\KeywordTok{using} \KeywordTok{namespace}\NormalTok{ Rcpp;}

\CommentTok{// [[Rcpp::export]]}
\NormalTok{NumericVector test(NumericVector v) \{}
\NormalTok{  v.sort();}
  \ControlFlowTok{return}\NormalTok{ v;}
\NormalTok{\}}
\end{Highlighting}
\end{Shaded}

\begin{Shaded}
\begin{Highlighting}[]
\OperatorTok{>}\StringTok{ }\KeywordTok{sourceCpp}\NormalTok{(}\StringTok{"test.cpp"}\NormalTok{)}

\OperatorTok{>}\StringTok{ }\KeywordTok{test}\NormalTok{(}\KeywordTok{c}\NormalTok{(}\DecValTok{5}\NormalTok{,}\DecValTok{4}\NormalTok{,}\DecValTok{7}\NormalTok{,}\DecValTok{6}\NormalTok{,}\DecValTok{8}\NormalTok{))}
\NormalTok{[}\DecValTok{1}\NormalTok{] }\DecValTok{4} \DecValTok{5} \DecValTok{6} \DecValTok{7} \DecValTok{8}
\end{Highlighting}
\end{Shaded}

\subsection{assign( first\_it, last\_it
)}\label{assign-first_it-last_it}

assign values specified by the iterator \texttt{first\_it} and
\texttt{last\_it} to this vector object.
将迭代器\texttt{first\_it}至\texttt{lates\_it}所指向的元素赋给向量对象。

举例说明,在\texttt{test.cpp}文件中键入以下代码,

\begin{Shaded}
\begin{Highlighting}[]
\PreprocessorTok{#include }\ImportTok{<Rcpp.h>}
\KeywordTok{using} \KeywordTok{namespace}\NormalTok{ Rcpp;}

\CommentTok{// [[Rcpp::export]]}
\NormalTok{NumericVector test(NumericVector v) \{}
\NormalTok{  NumericVector v1;}
\NormalTok{  v1.assign(v.begin(),v.end());}
  \ControlFlowTok{return}\NormalTok{ v1;}
\NormalTok{\}}
\end{Highlighting}
\end{Shaded}

\texttt{begin}和\texttt{end}也是成员函数,下面\ref{vector-begin}也有对应的例子

\begin{Shaded}
\begin{Highlighting}[]
\OperatorTok{>}\StringTok{ }\KeywordTok{sourceCpp}\NormalTok{(}\StringTok{"test.cpp"}\NormalTok{)}

\OperatorTok{>}\StringTok{ }\KeywordTok{test}\NormalTok{(}\DecValTok{1}\OperatorTok{:}\DecValTok{10}\NormalTok{)}
\NormalTok{ [}\DecValTok{1}\NormalTok{]  }\DecValTok{1}  \DecValTok{2}  \DecValTok{3}  \DecValTok{4}  \DecValTok{5}  \DecValTok{6}  \DecValTok{7}  \DecValTok{8}  \DecValTok{9} \DecValTok{10}
\end{Highlighting}
\end{Shaded}

\subsection{push\_back( x )}\label{push-backx}

在向量对象的最后加入新的标量值 \texttt{x}。

在\texttt{test.cpp}文件中键入以下代码,

\begin{Shaded}
\begin{Highlighting}[]
\PreprocessorTok{#include }\ImportTok{<Rcpp.h>}
\KeywordTok{using} \KeywordTok{namespace}\NormalTok{ Rcpp;}

\CommentTok{// [[Rcpp::export]]}
\NormalTok{NumericVector test(NumericVector v, }\DataTypeTok{double}\NormalTok{ var_PB) \{}
  
\NormalTok{  v.push_back(var_PB);}
  \ControlFlowTok{return}\NormalTok{ v;}
\NormalTok{\}}
\end{Highlighting}
\end{Shaded}

\begin{Shaded}
\begin{Highlighting}[]
\OperatorTok{>}\StringTok{ }\KeywordTok{sourceCpp}\NormalTok{(}\StringTok{"test.cpp"}\NormalTok{)}

\OperatorTok{>}\StringTok{ }\KeywordTok{test}\NormalTok{(}\KeywordTok{c}\NormalTok{(}\DecValTok{1}\NormalTok{,}\DecValTok{2}\NormalTok{,}\DecValTok{3}\NormalTok{,}\DecValTok{4}\NormalTok{,}\DecValTok{5}\NormalTok{),}\DecValTok{6}\NormalTok{)}
\NormalTok{[}\DecValTok{1}\NormalTok{] }\DecValTok{1} \DecValTok{2} \DecValTok{3} \DecValTok{4} \DecValTok{5} \DecValTok{6}
\end{Highlighting}
\end{Shaded}

\subsection{push\_back( x, name )}\label{push-back-name}

在向量后加入标量元素\texttt{x}时,指定其元素名称。

在\texttt{test.cpp}文件中键入以下代码,

\begin{Shaded}
\begin{Highlighting}[]
\PreprocessorTok{#include }\ImportTok{<Rcpp.h>}
\KeywordTok{using} \KeywordTok{namespace}\NormalTok{ Rcpp;}

\CommentTok{// [[Rcpp::export]]}
\NormalTok{NumericVector test(NumericVector v, }
                   \DataTypeTok{double}\NormalTok{ var_PB, }
                   \BuiltInTok{std::}\NormalTok{string var_Name) \{}
  
  \CommentTok{//v.names() = CharacterVector::create("a","b","c","d","e");}
\NormalTok{  v.push_back(var_PB,var_Name);}
  \ControlFlowTok{return}\NormalTok{ v;}
\NormalTok{\}}
\end{Highlighting}
\end{Shaded}

\begin{Shaded}
\begin{Highlighting}[]
\OperatorTok{>}\StringTok{ }\KeywordTok{sourceCpp}\NormalTok{(}\StringTok{'test.cpp'}\NormalTok{)}

\OperatorTok{>}\StringTok{ }\KeywordTok{test}\NormalTok{(}\KeywordTok{c}\NormalTok{(}\DataTypeTok{a =} \DecValTok{1}\NormalTok{,}\DataTypeTok{b =} \DecValTok{2}\NormalTok{,}\DataTypeTok{c =} \DecValTok{3}\NormalTok{,}\DataTypeTok{d =} \DecValTok{4}\NormalTok{,}\DataTypeTok{e =} \DecValTok{5}\NormalTok{),}\DecValTok{6}\NormalTok{,}\StringTok{"f"}\NormalTok{)}
\NormalTok{a b c d e f}
\DecValTok{1} \DecValTok{2} \DecValTok{3} \DecValTok{4} \DecValTok{5} \DecValTok{6}
\end{Highlighting}
\end{Shaded}

\subsection{push\_front( x )}\label{push-front-x}

在向量前面加入一个标量\texttt{x}。

在\texttt{test.cpp}文件中键入以下代码,

\begin{Shaded}
\begin{Highlighting}[]
\PreprocessorTok{#include }\ImportTok{<Rcpp.h>}
\KeywordTok{using} \KeywordTok{namespace}\NormalTok{ Rcpp;}

\CommentTok{// [[Rcpp::export]]}
\NormalTok{NumericVector test(NumericVector v, }
                   \DataTypeTok{double}\NormalTok{ var_f) \{}
\NormalTok{  v.push_front(var_f);}
  \ControlFlowTok{return}\NormalTok{ v;}
\NormalTok{\}}
\end{Highlighting}
\end{Shaded}

\begin{Shaded}
\begin{Highlighting}[]
\OperatorTok{>}\StringTok{ }\KeywordTok{sourceCpp}\NormalTok{(}\StringTok{'test.cpp'}\NormalTok{)}

\OperatorTok{>}\StringTok{ }\KeywordTok{test}\NormalTok{(}\KeywordTok{c}\NormalTok{(}\DecValTok{1}\NormalTok{,}\DecValTok{2}\NormalTok{,}\DecValTok{3}\NormalTok{,}\DecValTok{4}\NormalTok{,}\DecValTok{5}\NormalTok{),}\DecValTok{6}\NormalTok{)}
\DecValTok{6} \DecValTok{1} \DecValTok{2} \DecValTok{3} \DecValTok{4} \DecValTok{5} 
\end{Highlighting}
\end{Shaded}

\subsection{push\_front( x, name )}\label{push-front-name}

在向量前加入标量元素\texttt{x}时,指定其元素名称。

在\texttt{test.cpp}文件中键入以下代码,

\begin{Shaded}
\begin{Highlighting}[]
\PreprocessorTok{#include }\ImportTok{<Rcpp.h>}
\KeywordTok{using} \KeywordTok{namespace}\NormalTok{ Rcpp;}

\CommentTok{// [[Rcpp::export]]}
\NormalTok{NumericVector test(NumericVector v, }
                   \DataTypeTok{double}\NormalTok{ var_f, }
                   \BuiltInTok{std::}\NormalTok{string var_Name) \{}
  
  \CommentTok{//v.names() = CharacterVector::create("a","b","c","d","e");}
\NormalTok{  v.push_front(var_f,var_Name);}
  \ControlFlowTok{return}\NormalTok{ v;}
\NormalTok{\}}
\end{Highlighting}
\end{Shaded}

\begin{Shaded}
\begin{Highlighting}[]
\OperatorTok{>}\StringTok{ }\KeywordTok{sourceCpp}\NormalTok{(}\StringTok{'test.cpp'}\NormalTok{)}

\OperatorTok{>}\StringTok{ }\KeywordTok{test}\NormalTok{(}\KeywordTok{c}\NormalTok{(}\DataTypeTok{a =} \DecValTok{1}\NormalTok{,}\DataTypeTok{b =} \DecValTok{2}\NormalTok{,}\DataTypeTok{c =} \DecValTok{3}\NormalTok{,}\DataTypeTok{d =} \DecValTok{4}\NormalTok{,}\DataTypeTok{e =} \DecValTok{5}\NormalTok{),}\DecValTok{6}\NormalTok{,}\StringTok{"f"}\NormalTok{)}
\NormalTok{f a b c d e }
\DecValTok{6} \DecValTok{1} \DecValTok{2} \DecValTok{3} \DecValTok{4} \DecValTok{5} 
\end{Highlighting}
\end{Shaded}

\subsection{begin()}\label{vector-begin}

返回一个指向向量第一个元素的迭代器。

\subsection{end()}\label{vector-end}

返回一个指向向量最后一个元素的迭代器。 (\textbf{one past the last
element of this vector}).

以求和函数说明\texttt{begin()}和\texttt{end()}的作用。

\begin{Shaded}
\begin{Highlighting}[]
\PreprocessorTok{#include }\ImportTok{<Rcpp.h>}
\KeywordTok{using} \KeywordTok{namespace}\NormalTok{ Rcpp;}

\CommentTok{// [[Rcpp::export]]}
\DataTypeTok{double}\NormalTok{ test(NumericVector v) \{}
  
  \DataTypeTok{double}\NormalTok{ mysum = }\DecValTok{0}\NormalTok{;}
\NormalTok{  NumericVector::iterator it;}
  \ControlFlowTok{for}\NormalTok{(it = v.begin();it!=v.end();it++)\{}
\NormalTok{    mysum += *it;}
\NormalTok{  \}}
  \ControlFlowTok{return}\NormalTok{ mysum;}
\NormalTok{\}}
\end{Highlighting}
\end{Shaded}

\begin{Shaded}
\begin{Highlighting}[]
\OperatorTok{>}\StringTok{ }\KeywordTok{sourceCpp}\NormalTok{(}\StringTok{'test.cpp'}\NormalTok{)}

\OperatorTok{>}\StringTok{ }\KeywordTok{test}\NormalTok{(}\DecValTok{1}\OperatorTok{:}\DecValTok{10}\NormalTok{)}
\NormalTok{[}\DecValTok{1}\NormalTok{] }\DecValTok{55}
\end{Highlighting}
\end{Shaded}

\begin{quote}
在循环体内,我们用*it获取向量v中的元素,在指明循环范围的时候,也并不是我们熟悉的
int i = 0; i \textless{} n; i++
\end{quote}

该例子来源于Advanced
R中案例,请点击\href{http://adv-r.had.co.nz/Rcpp.html\#rcpp-classes}{传送门}。

\subsection{cbegin()}\label{cbegin}

返回一个指向向量第一个元素的具有const属性的迭代器。

\begin{quote}
无法用于元素的修改
\end{quote}

\subsection{cend()}\label{cend}

返回一个指向向量最后一个元素的具有const属性的迭代器。 (\textbf{one past
the last element of this vector}).

以求和函数说明\texttt{cbegin()}和\texttt{cend()}的作用。\textbf{下面的例子只在声明迭代器it的时候,将iterator改为const\_iterator},因为\texttt{cbegin()}和\texttt{cend()}得到是const\_iterator.

\begin{Shaded}
\begin{Highlighting}[]
\PreprocessorTok{#include }\ImportTok{<Rcpp.h>}
\KeywordTok{using} \KeywordTok{namespace}\NormalTok{ Rcpp;}

\CommentTok{// [[Rcpp::export]]}
\DataTypeTok{double}\NormalTok{ test(NumericVector v) \{}
  
  \DataTypeTok{double}\NormalTok{ mysum = }\DecValTok{0}\NormalTok{;}
\NormalTok{  NumericVector::const_iterator it;}
  \ControlFlowTok{for}\NormalTok{(it = v.begin();it!=v.end();it++)\{}
\NormalTok{    mysum += *it;}
\NormalTok{  \}}
  \ControlFlowTok{return}\NormalTok{ mysum;}
\NormalTok{\}}
\end{Highlighting}
\end{Shaded}

\begin{quote}
当然,对于c++不熟悉的用户,完全可以忽视\texttt{const\_iterator}。不声明it,而是采用auto,如下。
\end{quote}

\begin{Shaded}
\begin{Highlighting}[]
\PreprocessorTok{#include }\ImportTok{<Rcpp.h>}
\KeywordTok{using} \KeywordTok{namespace}\NormalTok{ Rcpp;}

\CommentTok{// [[Rcpp::plugins(cpp11)]]}
\CommentTok{// [[Rcpp::export]]}
\DataTypeTok{double}\NormalTok{ test(NumericVector v) \{}
  
  \DataTypeTok{double}\NormalTok{ mysum = }\DecValTok{0}\NormalTok{;}
  \ControlFlowTok{for}\NormalTok{(}\KeywordTok{auto}\NormalTok{ it = v.cbegin();it!=v.cend();it++)\{}
\NormalTok{    mysum += *it;}
    
\NormalTok{  \}}
  \ControlFlowTok{return}\NormalTok{ mysum;}
\NormalTok{\}}
\end{Highlighting}
\end{Shaded}

上面的这段代码,在循环体中,使用auto,来自动判别it的类型。对于不熟悉C++的用户而言(也包括我),是十分便捷的。但需要注意的是,一定要加上\texttt{//\ {[}{[}Rcpp::plugins(cpp11){]}{]}},表明你希望使用c++11的新特性,否则程序会报错。

\begin{Shaded}
\begin{Highlighting}[]
\OperatorTok{>}\StringTok{ }\KeywordTok{sourceCpp}\NormalTok{(}\StringTok{'test.cpp'}\NormalTok{)}

\OperatorTok{>}\StringTok{ }\KeywordTok{test}\NormalTok{(}\DecValTok{1}\OperatorTok{:}\DecValTok{10}\NormalTok{)}
\NormalTok{[}\DecValTok{1}\NormalTok{] }\DecValTok{55}
\end{Highlighting}
\end{Shaded}

\subsection{insert( i, x )}\label{inserti}

在数值索引\texttt{i}指定的位置插入标量\texttt{x}。返回一个指向插入元素的迭代器。

test.cpp文件如下:

\begin{Shaded}
\begin{Highlighting}[]
\PreprocessorTok{#include }\ImportTok{<Rcpp.h>}
\KeywordTok{using} \KeywordTok{namespace}\NormalTok{ Rcpp;}

\CommentTok{// [[Rcpp::export]]}
\NormalTok{NumericVector test(NumericVector v) \{}
\NormalTok{  v.insert(}\DecValTok{1}\NormalTok{,}\DecValTok{6}\NormalTok{);}
  \ControlFlowTok{return}\NormalTok{ v;}
\NormalTok{\}}
\end{Highlighting}
\end{Shaded}

R运行结果如下:

\begin{Shaded}
\begin{Highlighting}[]
\OperatorTok{>}\StringTok{ }\KeywordTok{sourceCpp}\NormalTok{(}\StringTok{"test.cpp"}\NormalTok{)}

\OperatorTok{>}\StringTok{ }\KeywordTok{test}\NormalTok{(}\DecValTok{2}\OperatorTok{:}\DecValTok{5}\NormalTok{)}
\NormalTok{[}\DecValTok{1}\NormalTok{] }\DecValTok{2} \DecValTok{6} \DecValTok{3} \DecValTok{4} \DecValTok{5}
\end{Highlighting}
\end{Shaded}

\subsection{insert( it, x )}\label{insertit}

在迭代器\texttt{it}指定的位置插入标量\texttt{x}。返回迭代器指向的元素。

test.cpp文件如下:

\begin{Shaded}
\begin{Highlighting}[]
\PreprocessorTok{#include }\ImportTok{<Rcpp.h>}
\KeywordTok{using} \KeywordTok{namespace}\NormalTok{ Rcpp;}

\CommentTok{// [[Rcpp::export]]}
\NormalTok{NumericVector test(NumericVector v) \{}
\NormalTok{  v.insert(v.begin()}\DecValTok{+1}\NormalTok{,}\DecValTok{6}\NormalTok{);}
  \ControlFlowTok{return}\NormalTok{ v;}
\NormalTok{\}}
\end{Highlighting}
\end{Shaded}

R运行结果如下:

\begin{Shaded}
\begin{Highlighting}[]
\OperatorTok{>}\StringTok{ }\KeywordTok{sourceCpp}\NormalTok{(}\StringTok{"test.cpp"}\NormalTok{)}

\OperatorTok{>}\StringTok{ }\KeywordTok{test}\NormalTok{(}\DecValTok{2}\OperatorTok{:}\DecValTok{5}\NormalTok{)}
\NormalTok{[}\DecValTok{1}\NormalTok{] }\DecValTok{2} \DecValTok{6} \DecValTok{3} \DecValTok{4} \DecValTok{5}
\end{Highlighting}
\end{Shaded}

\subsection{erase(i)}\label{vector-erasei}

擦除数值索引\texttt{i}指定的标量元素\texttt{x}。返回指向擦除元素之后一个元素的迭代器。

test.cpp文件如下:

\begin{Shaded}
\begin{Highlighting}[]
\PreprocessorTok{#include }\ImportTok{<Rcpp.h>}
\KeywordTok{using} \KeywordTok{namespace}\NormalTok{ Rcpp;}

\CommentTok{// [[Rcpp::export]]}
\DataTypeTok{double}\NormalTok{ test(NumericVector v) \{}
\NormalTok{  NumericVector::iterator it = v.erase(}\DecValTok{0}\NormalTok{);}
  \ControlFlowTok{return}\NormalTok{ *it;}
\NormalTok{\}}
\end{Highlighting}
\end{Shaded}

R运行结果如下:

\begin{Shaded}
\begin{Highlighting}[]
\OperatorTok{>}\StringTok{ }\KeywordTok{sourceCpp}\NormalTok{(}\StringTok{"test.cpp"}\NormalTok{)}

\OperatorTok{>}\StringTok{ }\KeywordTok{test}\NormalTok{(}\DecValTok{1}\OperatorTok{:}\DecValTok{5}\NormalTok{)}
\NormalTok{[}\DecValTok{1}\NormalTok{] }\DecValTok{2}
\end{Highlighting}
\end{Shaded}

\subsection{erase(it)}\label{vector-eraseit}

擦除迭代器\texttt{it}指向的元素。返回指向擦除元素之后一个元素的迭代器。

test.cpp文件如下:

\begin{Shaded}
\begin{Highlighting}[]
\PreprocessorTok{#include }\ImportTok{<Rcpp.h>}
\KeywordTok{using} \KeywordTok{namespace}\NormalTok{ Rcpp;}

\CommentTok{// [[Rcpp::export]]}
\DataTypeTok{double}\NormalTok{ test(NumericVector v) \{}
\NormalTok{  NumericVector::iterator it = v.erase(v.begin());}
  \ControlFlowTok{return}\NormalTok{ *it;}
\NormalTok{\}}
\end{Highlighting}
\end{Shaded}

R运行结果如下:

\begin{Shaded}
\begin{Highlighting}[]
\OperatorTok{>}\StringTok{ }\KeywordTok{sourceCpp}\NormalTok{(}\StringTok{"test.cpp"}\NormalTok{)}

\OperatorTok{>}\StringTok{ }\KeywordTok{test}\NormalTok{(}\DecValTok{1}\OperatorTok{:}\DecValTok{5}\NormalTok{)}
\NormalTok{[}\DecValTok{1}\NormalTok{] }\DecValTok{2}
\end{Highlighting}
\end{Shaded}

\subsection{erase( first\_i, last\_i )}\label{erase-first-lasti}

擦除数值索引\texttt{first\_i}至\texttt{last\_i\ -\ 1}之间的所有元素。返回指向擦除元素之后一个元素的迭代器。

test.cpp文件如下:

\begin{Shaded}
\begin{Highlighting}[]
\PreprocessorTok{#include }\ImportTok{<Rcpp.h>}
\KeywordTok{using} \KeywordTok{namespace}\NormalTok{ Rcpp;}

\CommentTok{// [[Rcpp::export]]}
\DataTypeTok{double}\NormalTok{ test(NumericVector v) \{}
\NormalTok{  NumericVector::iterator it = v.erase(}\DecValTok{0}\NormalTok{,}\DecValTok{3}\NormalTok{);}
  \ControlFlowTok{return}\NormalTok{ *it;}
\NormalTok{\}}
\end{Highlighting}
\end{Shaded}

R运行结果如下:

\begin{Shaded}
\begin{Highlighting}[]
\OperatorTok{>}\StringTok{ }\KeywordTok{sourceCpp}\NormalTok{(}\StringTok{"test.cpp"}\NormalTok{)}

\OperatorTok{>}\StringTok{ }\KeywordTok{test}\NormalTok{(}\DecValTok{1}\OperatorTok{:}\DecValTok{5}\NormalTok{)}
\NormalTok{[}\DecValTok{1}\NormalTok{] }\DecValTok{4}
\end{Highlighting}
\end{Shaded}

由于擦除的是索引0和3-1,即,第1个元素至第3个元素被擦除,返回的是对应原本第四个元素的迭代器,*it为4,也印证了结果。

\subsection{erase( first\_it, last\_it )}\label{erase-first-lastit}

擦除迭代器\texttt{first\_it}至\texttt{last\_it\ -\ 1}之间的所有元素。返回指向擦除元素之后一个元素的迭代器。

test.cpp文件如下:

\begin{Shaded}
\begin{Highlighting}[]
\PreprocessorTok{#include }\ImportTok{<Rcpp.h>}
\KeywordTok{using} \KeywordTok{namespace}\NormalTok{ Rcpp;}

\CommentTok{// [[Rcpp::export]]}
\DataTypeTok{double}\NormalTok{ test(NumericVector v) \{}
\NormalTok{  NumericVector::iterator it = v.erase(v.begin(),v.end());}
  \ControlFlowTok{return}\NormalTok{ *it;}
\NormalTok{\}}
\end{Highlighting}
\end{Shaded}

R运行结果如下:

\begin{Shaded}
\begin{Highlighting}[]
\OperatorTok{>}\StringTok{ }\KeywordTok{sourceCpp}\NormalTok{(}\StringTok{"test.cpp"}\NormalTok{)}

\OperatorTok{>}\StringTok{ }\KeywordTok{test}\NormalTok{(}\DecValTok{1}\OperatorTok{:}\DecValTok{5}\NormalTok{)}
\NormalTok{[}\DecValTok{1}\NormalTok{] }\DecValTok{5}
\end{Highlighting}
\end{Shaded}

\subsection{containsElementNamed(name)}\label{containsEN}

如果向量包含有某一个元素,其名称与字符串name相同,那么返回\texttt{true}。

test.cpp文件如下:

\begin{Shaded}
\begin{Highlighting}[]
\PreprocessorTok{#include }\ImportTok{<Rcpp.h>}
\KeywordTok{using} \KeywordTok{namespace}\NormalTok{ Rcpp;}

\CommentTok{// [[Rcpp::export]]}
\DataTypeTok{bool}\NormalTok{ test(NumericVector v) \{}
    \ControlFlowTok{if}\NormalTok{(v.containsElementNamed(}\StringTok{"b"}\NormalTok{))\{}
\NormalTok{        Rcout <<}\StringTok{"name is included"}\NormalTok{ << }\StringTok{'}\SpecialCharTok{\textbackslash{}n}\StringTok{'}\NormalTok{;}
\NormalTok{    \}}\ControlFlowTok{else}\NormalTok{\{}
\NormalTok{        Rcout <<}\StringTok{"name is not included"}\NormalTok{ << }\StringTok{'}\SpecialCharTok{\textbackslash{}n}\StringTok{'}\NormalTok{;}
\NormalTok{    \}}
  \ControlFlowTok{return}\NormalTok{ v.containsElementNamed(}\StringTok{"b"}\NormalTok{);}
\NormalTok{\}}
\end{Highlighting}
\end{Shaded}

R运行结果如下:

\begin{Shaded}
\begin{Highlighting}[]
\OperatorTok{>}\StringTok{ }\KeywordTok{sourceCpp}\NormalTok{(}\StringTok{"test.cpp"}\NormalTok{)}

\OperatorTok{>}\StringTok{ }\KeywordTok{test}\NormalTok{(}\KeywordTok{c}\NormalTok{(}\DataTypeTok{a =} \DecValTok{1}\NormalTok{, }\DataTypeTok{b =} \DecValTok{2}\NormalTok{, }\DataTypeTok{c =} \DecValTok{3}\NormalTok{))}
\NormalTok{name is included}
\NormalTok{[}\DecValTok{1}\NormalTok{] }\OtherTok{TRUE}
\end{Highlighting}
\end{Shaded}

\section{静态成员函数}\label{Static-member-functions}

静态成员函数是对象所在类的函数。k可以按照
\texttt{NumericVector::create()} 的方式来调用该静态成员函数。

\subsection{get\_na()}\label{get_na}

返回\texttt{Vector}类中的\texttt{NA}值。

test.cpp文件如下:

\begin{Shaded}
\begin{Highlighting}[]
\PreprocessorTok{#include }\ImportTok{<Rcpp.h>}
\KeywordTok{using} \KeywordTok{namespace}\NormalTok{ Rcpp;}

\CommentTok{// [[Rcpp::export]]}
\NormalTok{NumericVector test(NumericVector v) \{}
\NormalTok{  v.fill(NumericVector::get_na());}
  \ControlFlowTok{return}\NormalTok{ v;}
\NormalTok{\}}
\end{Highlighting}
\end{Shaded}

R运行结果如下:

\begin{Shaded}
\begin{Highlighting}[]
\OperatorTok{>}\StringTok{ }\KeywordTok{sourceCpp}\NormalTok{(}\StringTok{"test.cpp"}\NormalTok{)}

\OperatorTok{>}\StringTok{ }\KeywordTok{test}\NormalTok{(}\DecValTok{1}\OperatorTok{:}\DecValTok{5}\NormalTok{)}
\NormalTok{[}\DecValTok{1}\NormalTok{] }\OtherTok{NA} \OtherTok{NA} \OtherTok{NA} \OtherTok{NA} \OtherTok{NA}
\end{Highlighting}
\end{Shaded}

该例子有参考stackoverflow上的答案,详情点击\href{https://stackoverflow.com/questions/23748572/initializing-a-matrix-to-na-in-rcpp}{传送门}。

\subsection{is\_na(x)}\label{is_na}

如果\texttt{x}为\texttt{NA},则返回\texttt{true}。

test.cpp文件如下:

\begin{Shaded}
\begin{Highlighting}[]
\PreprocessorTok{#include }\ImportTok{<Rcpp.h>}
\KeywordTok{using} \KeywordTok{namespace}\NormalTok{ Rcpp;}

\CommentTok{// [[Rcpp::export]]}
\DataTypeTok{bool}\NormalTok{ test(NumericVector v,}\DataTypeTok{int}\NormalTok{ id) \{}
  \ControlFlowTok{return}\NormalTok{ NumericVector::is_na(v(id));}
\NormalTok{\}}
\end{Highlighting}
\end{Shaded}

R运行结果如下:

\begin{Shaded}
\begin{Highlighting}[]
\OperatorTok{>}\StringTok{ }\KeywordTok{sourceCpp}\NormalTok{(}\StringTok{"test.cpp"}\NormalTok{)}

\OperatorTok{>}\StringTok{   }\KeywordTok{test}\NormalTok{(}\KeywordTok{c}\NormalTok{(}\DecValTok{1}\OperatorTok{:}\DecValTok{3}\NormalTok{,}\OtherTok{NA}\NormalTok{,}\DecValTok{5}\NormalTok{),}\DecValTok{3}\NormalTok{)}
\NormalTok{[}\DecValTok{1}\NormalTok{] }\OtherTok{TRUE}
\end{Highlighting}
\end{Shaded}

\subsection{create( x1, x2, \ldots{})}\label{create}

创建一个\texttt{Vector}对象,其包含的元素由标量\texttt{x1},\texttt{x2}指定。参数最大个数为20。

可以命名元素或不命名,例子如下:

\begin{Shaded}
\begin{Highlighting}[]
\PreprocessorTok{#include }\ImportTok{<Rcpp.h>}
\KeywordTok{using} \KeywordTok{namespace}\NormalTok{ Rcpp;}

\CommentTok{// [[Rcpp::export]]}
\NormalTok{NumericVector test() \{}
  
\NormalTok{  NumericVector v = NumericVector::create(_[}\StringTok{"a"}\NormalTok{] = }\DecValTok{1}\NormalTok{,_[}\StringTok{"b"}\NormalTok{] = }\DecValTok{2}\NormalTok{,_[}\StringTok{"c"}\NormalTok{] = }\DecValTok{3}\NormalTok{,_[}\StringTok{"d"}\NormalTok{] = }\DecValTok{4}\NormalTok{);}
    \CommentTok{//NumericVector v = NumericVector::create(1,2,3,4);}
  \ControlFlowTok{return}\NormalTok{ v;}
\NormalTok{\}}
\end{Highlighting}
\end{Shaded}

\begin{Shaded}
\begin{Highlighting}[]
\OperatorTok{>}\StringTok{ }\KeywordTok{sourceCpp}\NormalTok{(}\StringTok{"test.cpp"}\NormalTok{)}

\OperatorTok{>}\StringTok{ }\KeywordTok{test}\NormalTok{()}
\NormalTok{a b c d }
\DecValTok{1} \DecValTok{2} \DecValTok{3} \DecValTok{4} 
\OperatorTok{>}\StringTok{ }
\end{Highlighting}
\end{Shaded}

\subsection{import( first\_it , last\_it )}\label{import}

创建一个\texttt{Vector}对象,其元素由迭代器\texttt{first\_it} 至
\texttt{last\_it\ -\ 1}指定。

\begin{Shaded}
\begin{Highlighting}[]
\PreprocessorTok{#include }\ImportTok{<Rcpp.h>}
\KeywordTok{using} \KeywordTok{namespace}\NormalTok{ Rcpp;}

\CommentTok{// [[Rcpp::export]]}
\NormalTok{NumericVector test(NumericVector v) \{}
\NormalTok{    NumericVector v1 = NumericVector::import(v.begin(),v.end());}
  \ControlFlowTok{return}\NormalTok{ v1;}
\NormalTok{\}}
\end{Highlighting}
\end{Shaded}

\begin{Shaded}
\begin{Highlighting}[]
\OperatorTok{>}\StringTok{ }\KeywordTok{sourceCpp}\NormalTok{(}\StringTok{"test.cpp"}\NormalTok{)}

\OperatorTok{>}\StringTok{ }\KeywordTok{test}\NormalTok{(}\DecValTok{1}\OperatorTok{:}\DecValTok{5}\NormalTok{)}
\NormalTok{[}\DecValTok{1}\NormalTok{] }\DecValTok{1} \DecValTok{2} \DecValTok{3} \DecValTok{4} \DecValTok{5}
\end{Highlighting}
\end{Shaded}

\subsection{import\_transform( first\_it, last\_it,
func)}\label{import_transform}

在\texttt{import(\ first\_it\ ,\ last\_it\ )}的基础上,对于每一个迭代器范围内的元素,进行\texttt{func}函数的操作。类似于\texttt{apply()}函数族。

\begin{Shaded}
\begin{Highlighting}[]
\PreprocessorTok{#include }\ImportTok{<Rcpp.h>}
\KeywordTok{using} \KeywordTok{namespace}\NormalTok{ Rcpp;}

\CommentTok{//构建mypow2函数,求取元素平方}

\CommentTok{// [[Rcpp::export]]}
\DataTypeTok{double}\NormalTok{ mypow2(}\DataTypeTok{double}\NormalTok{ x) \{}
  \ControlFlowTok{return}\NormalTok{ x*x;}
\NormalTok{\}}

\CommentTok{// [[Rcpp::export]]}
\NormalTok{NumericVector test(NumericVector v) \{}
\NormalTok{    NumericVector v1 = NumericVector::import_transform(v.begin(),v.end(),mypow2);}
  \ControlFlowTok{return}\NormalTok{ v1;}
\NormalTok{\}}
\end{Highlighting}
\end{Shaded}

\begin{Shaded}
\begin{Highlighting}[]
\OperatorTok{>}\StringTok{ }\KeywordTok{sourceCpp}\NormalTok{(}\StringTok{"test.cpp"}\NormalTok{)}

\OperatorTok{>}\StringTok{ }\KeywordTok{test}\NormalTok{(}\DecValTok{1}\OperatorTok{:}\DecValTok{5}\NormalTok{)}
\NormalTok{[}\DecValTok{1}\NormalTok{]  }\DecValTok{1}  \DecValTok{4}  \DecValTok{9} \DecValTok{16} \DecValTok{25}
\end{Highlighting}
\end{Shaded}

\chapter{Matrix类}\label{matrix}

\section{创建矩阵对象}\label{Creating-Matrix-object}

矩阵对象可由如下几种方式创建。

\begin{Shaded}
\begin{Highlighting}[]
\CommentTok{// 创建一个矩阵对象,等价于在R语句}
\CommentTok{// m <- matrix(0, nrow=2, ncol=2)}
\NormalTok{NumericMatrix m1( }\DecValTok{2}\NormalTok{ );}
\CommentTok{// m <- matrix(0, nrow=2, ncol=3)}
\NormalTok{NumericMatrix m2( }\DecValTok{2}\NormalTok{ , }\DecValTok{3}\NormalTok{ );}
\CommentTok{// m <- matrix(v, nrow=2, ncol=3)}
\NormalTok{NumericMatrix m3( }\DecValTok{2}\NormalTok{ , }\DecValTok{3}\NormalTok{ , v.begin() );}
\end{Highlighting}
\end{Shaded}

此外,R中的矩阵对象,实际上是行数和列数在属性\texttt{dim}中设定好的向量。因此,如果你在Rcpp中,创建一个有\texttt{dim}属性的向量,并且将其作为返回值传递给R,那么该向量在R中会被作为矩阵对待。

\begin{Shaded}
\begin{Highlighting}[]
\PreprocessorTok{#include }\ImportTok{<Rcpp.h>}
\KeywordTok{using} \KeywordTok{namespace}\NormalTok{ Rcpp;}
\CommentTok{// [[Rcpp::plugins(cpp11)]]}

\CommentTok{// [[Rcpp::export]]}
\NormalTok{NumericVector rcpp_matrix()\{}
    \CommentTok{// 创建一个向量对象}
\NormalTok{    NumericVector v = \{}\DecValTok{1}\NormalTok{,}\DecValTok{2}\NormalTok{,}\DecValTok{3}\NormalTok{,}\DecValTok{4}\NormalTok{\};}
    \CommentTok{// 设置该对象的`dim`属性}
\NormalTok{    v.attr(}\StringTok{"dim"}\NormalTok{) = Dimension(}\DecValTok{2}\NormalTok{, }\DecValTok{2}\NormalTok{);}
    \CommentTok{// 返回该对象给R}
    \ControlFlowTok{return}\NormalTok{ v;}
\NormalTok{\}}
\end{Highlighting}
\end{Shaded}

需要注意的是,c++98是不允许直接使用\texttt{v\ =\ \{1,2,3,4\}}来赋值的,因此,需要加上
\texttt{//\ {[}{[}Rcpp::plugins(cpp11){]}{]}},确保能使用C++11的新特性。

执行结果:

\begin{Shaded}
\begin{Highlighting}[]
\OperatorTok{>}\StringTok{ }\KeywordTok{rcpp_matrix}\NormalTok{()}
\NormalTok{     [,}\DecValTok{1}\NormalTok{] [,}\DecValTok{2}\NormalTok{]}
\NormalTok{[}\DecValTok{1}\NormalTok{,]    }\DecValTok{1}    \DecValTok{3}
\NormalTok{[}\DecValTok{2}\NormalTok{,]    }\DecValTok{2}    \DecValTok{4}
\end{Highlighting}
\end{Shaded}

然而,即便你给某个向量对象的\texttt{dim}属性赋值,在Rcpp中对象的类型还是为向量类。因此,如果你希望在Rcpp中,将其转化为矩阵类,你需要使用
\texttt{as\textless{}T\textgreater{}()}函数。

\begin{Shaded}
\begin{Highlighting}[]
\CommentTok{// 设定维度属性}
\NormalTok{v.attr(}\StringTok{"dim"}\NormalTok{) = Dimension(}\DecValTok{2}\NormalTok{, }\DecValTok{2}\NormalTok{);}
\CommentTok{// 转为矩阵类}
\NormalTok{NumericMatrix m = as<NumericMatrix>(v);}
\end{Highlighting}
\end{Shaded}

\section{访问矩阵元素}\label{Accessing-to-Matrix-elements}

通过使用\texttt{()}符号,你可以指定行,列号来获取,分配矩阵对象的元素值。和在向量中的索引类似,矩阵中的行与列号也是从0开始。如果你希望获取某一行或者一列,使用\texttt{\_}符号。也可以使用\texttt{{[}{]}}操作符,来访问矩阵元素(将矩阵理解为按列连接的向量)。

\begin{Shaded}
\begin{Highlighting}[]
\CommentTok{// 创建一个5*5的矩阵}
\NormalTok{NumericMatrix m( }\DecValTok{5}\NormalTok{, }\DecValTok{5}\NormalTok{ );}
\CommentTok{// 检索0行,2列的元素,即第一行,第三列的元素}
\DataTypeTok{double}\NormalTok{ x = m( }\DecValTok{0}\NormalTok{ , }\DecValTok{2}\NormalTok{ );}
\CommentTok{// 将0行(向量)赋给v}
\NormalTok{NumericVector v = m( }\DecValTok{0}\NormalTok{ , _ );}
\CommentTok{// 将2列赋给v}
\NormalTok{NumericVector v = m( _ , }\DecValTok{2}\NormalTok{ );}
\CommentTok{// 将矩阵m的0~1行,2~3列赋值给矩阵m2}
\NormalTok{NumericMatrix m2 = m( Range(}\DecValTok{0}\NormalTok{,}\DecValTok{1}\NormalTok{) , Range(}\DecValTok{2}\NormalTok{,}\DecValTok{3}\NormalTok{) );}
\CommentTok{// 按照向量的方式来检索矩阵元素}
\NormalTok{m[}\DecValTok{5}\NormalTok{]; }\CommentTok{// 指向m(0,1)的位置,即第6个元素(矩阵按列连接),为第1行,第2列元素,}
\end{Highlighting}
\end{Shaded}

\subsection{访问行,列与子矩阵}\label{row-col-sub}

Rcpp也提供了类型来进行矩阵特定部分的``引用''(references)。

\begin{Shaded}
\begin{Highlighting}[]
\NormalTok{NumericMatrix::Column col = m( _ , }\DecValTok{1}\NormalTok{);  }\CommentTok{// 对于列1的引用}
\NormalTok{NumericMatrix::Row    row = m( }\DecValTok{1}\NormalTok{ , _ ); }\CommentTok{// 对行1的引用}
\NormalTok{NumericMatrix::Sub    sub = m( Range(}\DecValTok{0}\NormalTok{,}\DecValTok{1}\NormalTok{) , Range(}\DecValTok{2}\NormalTok{,}\DecValTok{3}\NormalTok{) ); }\CommentTok{//对子矩阵的引用}
\end{Highlighting}
\end{Shaded}

对这些``引用''对象的赋值,\textbf{等效于}直接修改其原始矩阵的对应部分。比如,对于上面例子中的\texttt{col}对象进行赋值,会直接把m的列1的值进行对应的修改。

在\texttt{test.cpp}文件中键入下面的代码。

\begin{Shaded}
\begin{Highlighting}[]
\PreprocessorTok{#include }\ImportTok{<Rcpp.h>}
\KeywordTok{using} \KeywordTok{namespace}\NormalTok{ Rcpp;}

\CommentTok{// [[Rcpp::export]]}
\NormalTok{NumericMatrix test(NumericMatrix v, }\DataTypeTok{int}\NormalTok{ idx)\{}
\NormalTok{    NumericMatrix::Column col = v( _, idx);}
    \CommentTok{//将idx列的所有元素乘以2,v也会被修改}
\NormalTok{    col = col * }\DecValTok{2}\NormalTok{; }
    
    \CommentTok{//上行代码等效于}
    \CommentTok{//v( _, idx) = 2 * v( _, idx);}
    \ControlFlowTok{return}\NormalTok{ v;}
\NormalTok{\}}
\end{Highlighting}
\end{Shaded}

在R中的执行结果为:

\begin{Shaded}
\begin{Highlighting}[]
\OperatorTok{>}\StringTok{ }\KeywordTok{sourceCpp}\NormalTok{(}\StringTok{'test.cpp'}\NormalTok{)}

\OperatorTok{>}\StringTok{ }\NormalTok{a <-}\StringTok{ }\KeywordTok{matrix}\NormalTok{(}\DecValTok{1}\OperatorTok{:}\DecValTok{16}\NormalTok{,}\DecValTok{4}\NormalTok{,}\DecValTok{4}\NormalTok{)}

\OperatorTok{>}\StringTok{ }\NormalTok{a}
\NormalTok{     [,}\DecValTok{1}\NormalTok{] [,}\DecValTok{2}\NormalTok{] [,}\DecValTok{3}\NormalTok{] [,}\DecValTok{4}\NormalTok{]}
\NormalTok{[}\DecValTok{1}\NormalTok{,]    }\DecValTok{1}    \DecValTok{5}    \DecValTok{9}   \DecValTok{13}
\NormalTok{[}\DecValTok{2}\NormalTok{,]    }\DecValTok{2}    \DecValTok{6}   \DecValTok{10}   \DecValTok{14}
\NormalTok{[}\DecValTok{3}\NormalTok{,]    }\DecValTok{3}    \DecValTok{7}   \DecValTok{11}   \DecValTok{15}
\NormalTok{[}\DecValTok{4}\NormalTok{,]    }\DecValTok{4}    \DecValTok{8}   \DecValTok{12}   \DecValTok{16}

\OperatorTok{>}\StringTok{ }\KeywordTok{test}\NormalTok{(a,}\DecValTok{1}\NormalTok{)}
\NormalTok{     [,}\DecValTok{1}\NormalTok{] [,}\DecValTok{2}\NormalTok{] [,}\DecValTok{3}\NormalTok{] [,}\DecValTok{4}\NormalTok{]}
\NormalTok{[}\DecValTok{1}\NormalTok{,]    }\DecValTok{1}   \DecValTok{10}    \DecValTok{9}   \DecValTok{13}
\NormalTok{[}\DecValTok{2}\NormalTok{,]    }\DecValTok{2}   \DecValTok{12}   \DecValTok{10}   \DecValTok{14}
\NormalTok{[}\DecValTok{3}\NormalTok{,]    }\DecValTok{3}   \DecValTok{14}   \DecValTok{11}   \DecValTok{15}
\NormalTok{[}\DecValTok{4}\NormalTok{,]    }\DecValTok{4}   \DecValTok{16}   \DecValTok{12}   \DecValTok{16}
\end{Highlighting}
\end{Shaded}

可以看到,矩阵的第2列(Rcpp索引为1)已经变为原来的2倍。

\section{成员函数}\label{Member-functions-matrix}

Since \texttt{Matrix} is actually \texttt{Vector}, \texttt{Matrix}
basically has the same member functions as \texttt{Vector}. Thus, member
functions unique to \texttt{Matrix} are only presented below.

此前提到,\texttt{Matrix}实际上也是\texttt{Vector},所以,\texttt{Matrix}基本上与\texttt{Vector}有着相同的成员函数。因此,在在此列出\texttt{Matrix}自身独特的成员函数。

\subsection{nrow() rows()}\label{nrow-rows}

返回行数。

在\texttt{test.cpp}文件中键入下面的代码。

\begin{Shaded}
\begin{Highlighting}[]
\PreprocessorTok{#include }\ImportTok{<Rcpp.h>}
\KeywordTok{using} \KeywordTok{namespace}\NormalTok{ Rcpp;}

\CommentTok{// [[Rcpp::export]]}
\DataTypeTok{int}\NormalTok{ test(NumericMatrix v)\{}

    \ControlFlowTok{return}\NormalTok{ v.nrow();}
\NormalTok{\}}
\end{Highlighting}
\end{Shaded}

在R中的执行结果为:

\begin{Shaded}
\begin{Highlighting}[]
\OperatorTok{>}\StringTok{ }\KeywordTok{sourceCpp}\NormalTok{(}\StringTok{'test.cpp'}\NormalTok{)}

\OperatorTok{>}\StringTok{ }\KeywordTok{test}\NormalTok{(}\KeywordTok{matrix}\NormalTok{(}\DecValTok{1}\OperatorTok{:}\DecValTok{16}\NormalTok{,}\DecValTok{2}\NormalTok{,}\DecValTok{8}\NormalTok{))}
\NormalTok{[}\DecValTok{1}\NormalTok{] }\DecValTok{2}
\end{Highlighting}
\end{Shaded}

\subsection{ncol() cols()}\label{ncol-cols}

返回矩阵列数。

在\texttt{test.cpp}文件中键入下面的代码。

\begin{Shaded}
\begin{Highlighting}[]
\PreprocessorTok{#include }\ImportTok{<Rcpp.h>}
\KeywordTok{using} \KeywordTok{namespace}\NormalTok{ Rcpp;}

\CommentTok{// [[Rcpp::export]]}
\DataTypeTok{int}\NormalTok{ test(NumericMatrix v)\{}

    \ControlFlowTok{return}\NormalTok{ v.ncol();}
\NormalTok{\}}
\end{Highlighting}
\end{Shaded}

在R中的执行结果为:

\begin{Shaded}
\begin{Highlighting}[]
\OperatorTok{>}\StringTok{ }\KeywordTok{sourceCpp}\NormalTok{(}\StringTok{'test.cpp'}\NormalTok{)}

\OperatorTok{>}\StringTok{ }\KeywordTok{test}\NormalTok{(}\KeywordTok{matrix}\NormalTok{(}\DecValTok{1}\OperatorTok{:}\DecValTok{16}\NormalTok{,}\DecValTok{2}\NormalTok{,}\DecValTok{8}\NormalTok{))}
\NormalTok{[}\DecValTok{1}\NormalTok{] }\DecValTok{8}
\end{Highlighting}
\end{Shaded}

\subsection{row( i )}\label{rowi}

返回矩阵
行\texttt{i}的``引用'',关于``引用''的具体信息,可参考\ref{row-col-sub}。

在\texttt{test.cpp}文件中键入下面的代码。

\begin{Shaded}
\begin{Highlighting}[]
\PreprocessorTok{#include }\ImportTok{<Rcpp.h>}
\KeywordTok{using} \KeywordTok{namespace}\NormalTok{ Rcpp;}

\CommentTok{// [[Rcpp::export]]}
\NormalTok{NumericMatrix test(NumericMatrix v)\{}
\NormalTok{    v.row(}\DecValTok{1}\NormalTok{) = }\DecValTok{2}\NormalTok{ * v.row(}\DecValTok{1}\NormalTok{);}
    \ControlFlowTok{return}\NormalTok{ v;}
\NormalTok{\}}
\end{Highlighting}
\end{Shaded}

在R中的执行结果为:

\begin{Shaded}
\begin{Highlighting}[]
\OperatorTok{>}\StringTok{ }\KeywordTok{sourceCpp}\NormalTok{(}\StringTok{'test.cpp'}\NormalTok{)}

\OperatorTok{>}\StringTok{ }\NormalTok{(a <-}\StringTok{ }\KeywordTok{matrix}\NormalTok{(}\DecValTok{1}\OperatorTok{:}\DecValTok{8}\NormalTok{,}\DecValTok{2}\NormalTok{,}\DecValTok{4}\NormalTok{))}
\NormalTok{     [,}\DecValTok{1}\NormalTok{] [,}\DecValTok{2}\NormalTok{] [,}\DecValTok{3}\NormalTok{] [,}\DecValTok{4}\NormalTok{]}
\NormalTok{[}\DecValTok{1}\NormalTok{,]    }\DecValTok{1}    \DecValTok{3}    \DecValTok{5}    \DecValTok{7}
\NormalTok{[}\DecValTok{2}\NormalTok{,]    }\DecValTok{2}    \DecValTok{4}    \DecValTok{6}    \DecValTok{8}

\OperatorTok{>}\StringTok{ }\KeywordTok{test}\NormalTok{(a)}
\NormalTok{     [,}\DecValTok{1}\NormalTok{] [,}\DecValTok{2}\NormalTok{] [,}\DecValTok{3}\NormalTok{] [,}\DecValTok{4}\NormalTok{]}
\NormalTok{[}\DecValTok{1}\NormalTok{,]    }\DecValTok{1}    \DecValTok{3}    \DecValTok{5}    \DecValTok{7}
\NormalTok{[}\DecValTok{2}\NormalTok{,]    }\DecValTok{4}    \DecValTok{8}   \DecValTok{12}   \DecValTok{16}
\end{Highlighting}
\end{Shaded}

\subsection{column( i )}\label{columni}

返回矩阵 行\texttt{i}列的``引用''

\subsection{fill\_diag( x )}\label{fill_diag}

使用\texttt{x}填充矩阵对角线元素。

在\texttt{test.cpp}文件中键入下面的代码。

\begin{Shaded}
\begin{Highlighting}[]
\PreprocessorTok{#include }\ImportTok{<Rcpp.h>}
\KeywordTok{using} \KeywordTok{namespace}\NormalTok{ Rcpp;}

\CommentTok{// [[Rcpp::export]]}
\NormalTok{NumericMatrix test(NumericMatrix v,}\DataTypeTok{double}\NormalTok{ x)\{}
\NormalTok{    v.fill_diag(x);}
    \ControlFlowTok{return}\NormalTok{ v;}
\NormalTok{\}}
\end{Highlighting}
\end{Shaded}

在R中的执行结果为:

\begin{Shaded}
\begin{Highlighting}[]
\OperatorTok{>}\StringTok{ }\KeywordTok{sourceCpp}\NormalTok{(}\StringTok{'test.cpp'}\NormalTok{)}

\OperatorTok{>}\StringTok{ }\NormalTok{(a <-}\StringTok{ }\KeywordTok{matrix}\NormalTok{(}\DecValTok{0}\NormalTok{,}\DecValTok{4}\NormalTok{,}\DecValTok{4}\NormalTok{))}
\NormalTok{     [,}\DecValTok{1}\NormalTok{] [,}\DecValTok{2}\NormalTok{] [,}\DecValTok{3}\NormalTok{] [,}\DecValTok{4}\NormalTok{]}
\NormalTok{[}\DecValTok{1}\NormalTok{,]    }\DecValTok{0}    \DecValTok{0}    \DecValTok{0}    \DecValTok{0}
\NormalTok{[}\DecValTok{2}\NormalTok{,]    }\DecValTok{0}    \DecValTok{0}    \DecValTok{0}    \DecValTok{0}
\NormalTok{[}\DecValTok{3}\NormalTok{,]    }\DecValTok{0}    \DecValTok{0}    \DecValTok{0}    \DecValTok{0}
\NormalTok{[}\DecValTok{4}\NormalTok{,]    }\DecValTok{0}    \DecValTok{0}    \DecValTok{0}    \DecValTok{0}

\OperatorTok{>}\StringTok{ }\KeywordTok{test}\NormalTok{(a,}\DecValTok{1}\NormalTok{)}
\NormalTok{     [,}\DecValTok{1}\NormalTok{] [,}\DecValTok{2}\NormalTok{] [,}\DecValTok{3}\NormalTok{] [,}\DecValTok{4}\NormalTok{]}
\NormalTok{[}\DecValTok{1}\NormalTok{,]    }\DecValTok{1}    \DecValTok{0}    \DecValTok{0}    \DecValTok{0}
\NormalTok{[}\DecValTok{2}\NormalTok{,]    }\DecValTok{0}    \DecValTok{1}    \DecValTok{0}    \DecValTok{0}
\NormalTok{[}\DecValTok{3}\NormalTok{,]    }\DecValTok{0}    \DecValTok{0}    \DecValTok{1}    \DecValTok{0}
\NormalTok{[}\DecValTok{4}\NormalTok{,]    }\DecValTok{0}    \DecValTok{0}    \DecValTok{0}    \DecValTok{1}
\end{Highlighting}
\end{Shaded}

\subsection{offset( i, j )}\label{offsetij}

返回\texttt{i}行和\texttt{j}列对应的元素,在将矩阵作为列向量连接时其对应的索引。

\begin{quote}
matrix中的offset函数是私有的,似乎因此导致调用失败。
\end{quote}

\section{静态成员函数}\label{MStatic-member-functions}

\texttt{Matrix}基本上有着和\texttt{Vector}相同的成员函数。其独特的成员函数在此处列出。

\subsection{Matrix::diag( size, x )}\label{Mdiag}

返回一个矩阵,行列数均为\texttt{size},对角元素为\texttt{x}。

在\texttt{test.cpp}文件中键入下面的代码。

\begin{Shaded}
\begin{Highlighting}[]
\PreprocessorTok{#include }\ImportTok{<Rcpp.h>}
\KeywordTok{using} \KeywordTok{namespace}\NormalTok{ Rcpp;}

\CommentTok{// [[Rcpp::export]]}
\NormalTok{NumericMatrix test(}\DataTypeTok{int}\NormalTok{ size, }\DataTypeTok{double}\NormalTok{ x)\{}
\NormalTok{  NumericMatrix v = NumericMatrix::diag(size,x);}
  \ControlFlowTok{return}\NormalTok{ v;}
\NormalTok{\}}
\end{Highlighting}
\end{Shaded}

在R中的执行结果为:

\begin{Shaded}
\begin{Highlighting}[]
\OperatorTok{>}\StringTok{ }\KeywordTok{sourceCpp}\NormalTok{(}\StringTok{'test.cpp'}\NormalTok{)}

\OperatorTok{>}\StringTok{ }\KeywordTok{test}\NormalTok{(3L,}\DecValTok{1}\NormalTok{)}
\NormalTok{     [,}\DecValTok{1}\NormalTok{] [,}\DecValTok{2}\NormalTok{] [,}\DecValTok{3}\NormalTok{]}
\NormalTok{[}\DecValTok{1}\NormalTok{,]    }\DecValTok{1}    \DecValTok{0}    \DecValTok{0}
\NormalTok{[}\DecValTok{2}\NormalTok{,]    }\DecValTok{0}    \DecValTok{1}    \DecValTok{0}
\NormalTok{[}\DecValTok{3}\NormalTok{,]    }\DecValTok{0}    \DecValTok{0}    \DecValTok{1}
\end{Highlighting}
\end{Shaded}

\section{与Matrix相关的其他函数}\label{other-functions}

此部分展示一些其他与矩阵相关的函数。

\subsection{rownames( m )}\label{rownamesMatrix}

获取和设定矩阵行名。

\begin{Shaded}
\begin{Highlighting}[]
\NormalTok{CharacterVector ch = rownames(m);}
\NormalTok{rownames(m) = ch;}
\end{Highlighting}
\end{Shaded}

在\texttt{test.cpp}文件中键入下面的代码。

\begin{Shaded}
\begin{Highlighting}[]
\PreprocessorTok{#include }\ImportTok{<Rcpp.h>}
\KeywordTok{using} \KeywordTok{namespace}\NormalTok{ Rcpp;}

\CommentTok{// [[Rcpp::export]]}
\NormalTok{NumericMatrix test(NumericMatrix v1, NumericMatrix v2)\{}
\NormalTok{  CharacterVector v1_rname = rownames(v1);}
\NormalTok{  rownames(v2) = v1_rname;}
  \ControlFlowTok{return}\NormalTok{ v2;}
\NormalTok{\}}
\end{Highlighting}
\end{Shaded}

在R中的执行结果为:

\begin{Shaded}
\begin{Highlighting}[]
\OperatorTok{>}\StringTok{ }\KeywordTok{sourceCpp}\NormalTok{(}\StringTok{'test.cpp'}\NormalTok{)}

\OperatorTok{>}\StringTok{ }\NormalTok{a <-}\StringTok{ }\KeywordTok{matrix}\NormalTok{(}\DecValTok{0}\NormalTok{,}\DecValTok{3}\NormalTok{,}\DecValTok{3}\NormalTok{)}

\OperatorTok{>}\StringTok{ }\NormalTok{b <-}\StringTok{ }\KeywordTok{matrix}\NormalTok{(}\DecValTok{1}\NormalTok{,}\DecValTok{3}\NormalTok{,}\DecValTok{3}\NormalTok{)}

\OperatorTok{>}\StringTok{ }\KeywordTok{rownames}\NormalTok{(a) <-}\StringTok{ }\KeywordTok{c}\NormalTok{(}\StringTok{'a'}\NormalTok{,}\StringTok{'b'}\NormalTok{,}\StringTok{'c'}\NormalTok{)}

\OperatorTok{>}\StringTok{ }\NormalTok{a;b}
\NormalTok{  [,}\DecValTok{1}\NormalTok{] [,}\DecValTok{2}\NormalTok{] [,}\DecValTok{3}\NormalTok{]}
\NormalTok{a    }\DecValTok{0}    \DecValTok{0}    \DecValTok{0}
\NormalTok{b    }\DecValTok{0}    \DecValTok{0}    \DecValTok{0}
\NormalTok{c    }\DecValTok{0}    \DecValTok{0}    \DecValTok{0}
\NormalTok{     [,}\DecValTok{1}\NormalTok{] [,}\DecValTok{2}\NormalTok{] [,}\DecValTok{3}\NormalTok{]}
\NormalTok{[}\DecValTok{1}\NormalTok{,]    }\DecValTok{1}    \DecValTok{1}    \DecValTok{1}
\NormalTok{[}\DecValTok{2}\NormalTok{,]    }\DecValTok{1}    \DecValTok{1}    \DecValTok{1}
\NormalTok{[}\DecValTok{3}\NormalTok{,]    }\DecValTok{1}    \DecValTok{1}    \DecValTok{1}

\OperatorTok{>}\StringTok{ }\KeywordTok{test}\NormalTok{(a,b)}
\NormalTok{  [,}\DecValTok{1}\NormalTok{] [,}\DecValTok{2}\NormalTok{] [,}\DecValTok{3}\NormalTok{]}
\NormalTok{a    }\DecValTok{1}    \DecValTok{1}    \DecValTok{1}
\NormalTok{b    }\DecValTok{1}    \DecValTok{1}    \DecValTok{1}
\NormalTok{c    }\DecValTok{1}    \DecValTok{1}    \DecValTok{1}
\end{Highlighting}
\end{Shaded}

\subsection{colnames( m )}\label{colnamesMatrix}

获取和设定矩阵列名,方法同上。

\begin{verbatim}
CharacterVector ch = colnames(m);
colnames(m) = ch;
\end{verbatim}

\subsection{transpose( m )}\label{transpose}

返回矩阵\texttt{m}的转置。

在\texttt{test.cpp}文件中键入下面的代码。

\begin{Shaded}
\begin{Highlighting}[]
\PreprocessorTok{#include }\ImportTok{<Rcpp.h>}
\KeywordTok{using} \KeywordTok{namespace}\NormalTok{ Rcpp;}

\CommentTok{// [[Rcpp::export]]}
\NormalTok{NumericMatrix test(NumericMatrix v)\{}
  \ControlFlowTok{return}\NormalTok{ transpose(v);}
\NormalTok{\}}
\end{Highlighting}
\end{Shaded}

在R中的执行结果为:

\begin{Shaded}
\begin{Highlighting}[]
\OperatorTok{>}\StringTok{ }\KeywordTok{sourceCpp}\NormalTok{(}\StringTok{'test.cpp'}\NormalTok{)}

\OperatorTok{>}\StringTok{ }\NormalTok{(a <-}\StringTok{ }\KeywordTok{matrix}\NormalTok{(}\DecValTok{1}\OperatorTok{:}\DecValTok{9}\NormalTok{,}\DecValTok{3}\NormalTok{,}\DecValTok{3}\NormalTok{))}
\NormalTok{     [,}\DecValTok{1}\NormalTok{] [,}\DecValTok{2}\NormalTok{] [,}\DecValTok{3}\NormalTok{]}
\NormalTok{[}\DecValTok{1}\NormalTok{,]    }\DecValTok{1}    \DecValTok{4}    \DecValTok{7}
\NormalTok{[}\DecValTok{2}\NormalTok{,]    }\DecValTok{2}    \DecValTok{5}    \DecValTok{8}
\NormalTok{[}\DecValTok{3}\NormalTok{,]    }\DecValTok{3}    \DecValTok{6}    \DecValTok{9}

\OperatorTok{>}\StringTok{ }\KeywordTok{test}\NormalTok{(a)}
\NormalTok{     [,}\DecValTok{1}\NormalTok{] [,}\DecValTok{2}\NormalTok{] [,}\DecValTok{3}\NormalTok{]}
\NormalTok{[}\DecValTok{1}\NormalTok{,]    }\DecValTok{1}    \DecValTok{2}    \DecValTok{3}
\NormalTok{[}\DecValTok{2}\NormalTok{,]    }\DecValTok{4}    \DecValTok{5}    \DecValTok{6}
\NormalTok{[}\DecValTok{3}\NormalTok{,]    }\DecValTok{7}    \DecValTok{8}    \DecValTok{9}
\end{Highlighting}
\end{Shaded}

\chapter{向量运算}\label{Vector-operations}

\section{数学运算}\label{Arithmetic-operations}

使用\texttt{+\ -\ *\ /}运算符,用户可以对相同长度的向量进行元素级别的运算。

\begin{Shaded}
\begin{Highlighting}[]
\NormalTok{NumericVector x ;}
\NormalTok{NumericVector y ;}
\CommentTok{// 向量与向量运算}
\NormalTok{NumericVector res = x + y ;}
\NormalTok{NumericVector res = x - y ;}
\NormalTok{NumericVector res = x * y ;}
\NormalTok{NumericVector res = x / y ;}
\CommentTok{// 向量与标量运算}
\NormalTok{NumericVector res = x   + }\FloatTok{2.0}\NormalTok{ ;}
\NormalTok{NumericVector res = }\FloatTok{2.0}\NormalTok{ - x;}
\NormalTok{NumericVector res = y   * }\FloatTok{2.0}\NormalTok{ ;}
\NormalTok{NumericVector res = }\FloatTok{2.0}\NormalTok{ / y;}
\CommentTok{// expression and expression operation}
\NormalTok{NumericVector res = x * y + y / }\FloatTok{2.0}\NormalTok{ ;}
\NormalTok{NumericVector res = x * ( y - }\FloatTok{2.0}\NormalTok{ ) ;}
\NormalTok{NumericVector res = x / ( y * y ) ;}
\end{Highlighting}
\end{Shaded}

\texttt{-}号逆转了符号。

\begin{Shaded}
\begin{Highlighting}[]
\NormalTok{NumericVector res = -x ;}
\end{Highlighting}
\end{Shaded}

\section{比较运算}\label{Comparison-operations}

使用\texttt{==} \texttt{!\ =} \texttt{\textless{}}
\texttt{\textgreater{}} \texttt{\textgreater{}\ =}
\texttt{\textless{}=}运算符进行向量比较,会产生逻辑向量。用户可以通过逻辑向量来访问向量元素。

\begin{Shaded}
\begin{Highlighting}[]
\NormalTok{NumericVector x ;}
\NormalTok{NumericVector y ;}
\CommentTok{// Comparison of vector and vector}
\NormalTok{LogicalVector res = x < y ;}
\NormalTok{LogicalVector res = x > y ;}
\NormalTok{LogicalVector res = x <= y ;}
\NormalTok{LogicalVector res = x >= y ;}
\NormalTok{LogicalVector res = x == y ;}
\NormalTok{LogicalVector res = x != y ;}
\CommentTok{// Comparison of vector and scalar}
\NormalTok{LogicalVector res = x < }\DecValTok{2}\NormalTok{ ;}
\NormalTok{LogicalVector res = }\DecValTok{2}\NormalTok{ > x;}
\NormalTok{LogicalVector res = y <= }\DecValTok{2}\NormalTok{ ;}
\NormalTok{LogicalVector res = }\DecValTok{2}\NormalTok{ != y;}
\CommentTok{// Comparison of expression and expression}
\NormalTok{LogicalVector res = ( x + y ) < ( x*x ) ;}
\NormalTok{LogicalVector res = ( x + y ) >= ( x*x ) ;}
\NormalTok{LogicalVector res = ( x + y ) == ( x*x ) ;}
\end{Highlighting}
\end{Shaded}

\texttt{!} 表示对逻辑值的否运算。

\begin{Shaded}
\begin{Highlighting}[]
\NormalTok{LogicalVector res = ! ( x < y );}
\end{Highlighting}
\end{Shaded}

通过逻辑向量来访问向量元素。

\begin{Shaded}
\begin{Highlighting}[]
\NormalTok{NumericVector res = x[x < }\DecValTok{2}\NormalTok{];}
\end{Highlighting}
\end{Shaded}

\chapter{逻辑运算}\label{Logical-operations}

\section{LogicalVector}\label{LogicalVector}

\subsection{LogicalVector元素的数据类型}\label{LogicalVector-elements}

LogicalVector的元素类型\textbf{不是}\texttt{bool}。这是因为,\texttt{bool}只能表示\texttt{true}和\texttt{false},但是在R中,逻辑向量有三种可能的取值,即\texttt{TRUE},\texttt{FALSE}以及\texttt{NA}。因此,LogicalVector元素的数据类型为\texttt{int},而非\texttt{bool}。

在Rcpp中,\texttt{TRUE}用1表示,\texttt{FALSE}用0表示,而\texttt{NA}由\texttt{NA\_LOGICAL}表示(整型的最小值:-2147483648)。

\begin{longtable}[]{@{}cccc@{}}
\toprule
R & Rcpp & int & bool\tabularnewline
\midrule
\endhead
TRUE & TRUE & 1 (除却-2147483648至0之间的int) & true\tabularnewline
FALSE & FALSE & 0 & false\tabularnewline
NA & NA\_LOGICAL & -2147483648 & true\tabularnewline
\bottomrule
\end{longtable}

\section{逻辑运算符}\label{Logical-operations-subtitle}

使用运算符\texttt{\&}(与)\texttt{\textbar{}}(或)\texttt{!}(非)来对LogicalVector中的每个元素进行逻辑运算。

\begin{Shaded}
\begin{Highlighting}[]
\NormalTok{LogicalVector v1 = \{}\DecValTok{1}\NormalTok{,}\DecValTok{1}\NormalTok{,}\DecValTok{0}\NormalTok{,}\DecValTok{0}\NormalTok{\};}
\NormalTok{LogicalVector v2 = \{}\DecValTok{1}\NormalTok{,}\DecValTok{0}\NormalTok{,}\DecValTok{1}\NormalTok{,}\DecValTok{0}\NormalTok{\};}
\NormalTok{LogicalVector res1 = v1 & v2;}
\NormalTok{LogicalVector res2 = v1 | v2;}
\NormalTok{LogicalVector res3 = !(v1 | v2);}
\NormalTok{Rcout << res1 << }\StringTok{"}\SpecialCharTok{\textbackslash{}n}\StringTok{"}\NormalTok{; }\CommentTok{// 1 0 0 0}
\NormalTok{Rcout << res2 << }\StringTok{"}\SpecialCharTok{\textbackslash{}n}\StringTok{"}\NormalTok{; }\CommentTok{// 1 1 1 0}
\NormalTok{Rcout << res3 << }\StringTok{"}\SpecialCharTok{\textbackslash{}n}\StringTok{"}\NormalTok{; }\CommentTok{// 0 0 0 1}
\end{Highlighting}
\end{Shaded}

\section{接收LogicalVector的函数}\label{Function-that-receives-LogicalVector}

接收\texttt{LogicalVector}的函数有\texttt{all()}, \texttt{any()} 及
\texttt{ifelse()}等。

\subsection{all(), any()}\label{all-any}

对于\texttt{LogicalVector}
v,当所有元素都为\texttt{TRUE}时,\texttt{all\ (v)}返回\texttt{TRUE},当任意一个元素为\texttt{TRUE}时,\texttt{any\ (v)}返回\texttt{TRUE},

然而,用户并不能在\texttt{if}语句的条件表达式中,使用\texttt{all()}或者\texttt{any()}的返回值。这是因为这两者的返回值并不是\texttt{bool}型,而是\texttt{SingleLogicalResult}型。如果要在\texttt{if}条件语句中使用这两个函数,可以考虑使用\texttt{is\_true()},
\texttt{is\_false()} 和
\texttt{is\_na()}。这些函数会把\texttt{SingleLogicalResult}
类型转为\texttt{bool}型。

下面的代码展示了,如何在\texttt{if}语句的条件表达式中使用\texttt{all()}和\texttt{any()}。在这个例子中,条件表达式的值为\texttt{true},\texttt{all()}和\texttt{any()}的返回值也会被打印显示。

在\texttt{test.cpp}文件中输入下面代码。

\begin{Shaded}
\begin{Highlighting}[]
\CommentTok{// [[Rcpp::export]]}
\NormalTok{List rcpp_logical_03()\{}
\NormalTok{  LogicalVector v1 = LogicalVector::create(}\DecValTok{1}\NormalTok{,}\DecValTok{1}\NormalTok{,}\DecValTok{1}\NormalTok{,NA_LOGICAL);}
\NormalTok{  LogicalVector v2 = LogicalVector::create(}\DecValTok{0}\NormalTok{,}\DecValTok{1}\NormalTok{,}\DecValTok{0}\NormalTok{,NA_LOGICAL);}
  \CommentTok{// 对于包含有NA的Logical向量,all (), any () 函数的结果与R一致 }
\NormalTok{  LogicalVector lv1 = all( v1 );   }\CommentTok{// NA}
\NormalTok{  LogicalVector lv2 = all( v2 );   }\CommentTok{// FALSE}
\NormalTok{  LogicalVector lv3 = any( v2 ); }\CommentTok{// TRUE}
  \CommentTok{// 将`SingleLogicalResult` 类型转为`bool`型,然后赋值}
  \DataTypeTok{bool}\NormalTok{ b1 = is_true ( all(v1) );  }\CommentTok{// false}
  \DataTypeTok{bool}\NormalTok{ b2 = is_false( all(v1) );  }\CommentTok{// false}
  \DataTypeTok{bool}\NormalTok{ b3 = is_na   ( all(v1) );  }\CommentTok{// true}
  \CommentTok{// 在if语句条件判别式中的情况}
  \ControlFlowTok{if}\NormalTok{(is_na(all( v1 ))) \{ }\CommentTok{// OK}
\NormalTok{    Rcout << }\StringTok{"all( v1 ) is NA}\SpecialCharTok{\textbackslash{}n}\StringTok{"}\NormalTok{;}
\NormalTok{  \}}
  \CommentTok{//打印所有信息}
\NormalTok{  Rcout << }\StringTok{"lv1"}\NormalTok{ << lv1 << }\StringTok{'}\SpecialCharTok{\textbackslash{}n}\StringTok{'}\NormalTok{;}
\NormalTok{  Rcout << }\StringTok{"lv2"}\NormalTok{ << lv2 << }\StringTok{'}\SpecialCharTok{\textbackslash{}n}\StringTok{'}\NormalTok{;}
\NormalTok{  Rcout << }\StringTok{"lv3"}\NormalTok{ <<lv3 << }\StringTok{'}\SpecialCharTok{\textbackslash{}n}\StringTok{'}\NormalTok{;}
\NormalTok{  Rcout << }\StringTok{"b1: "}\NormalTok{ << b1 << }\StringTok{'}\SpecialCharTok{\textbackslash{}n}\StringTok{'}\NormalTok{;}
\NormalTok{  Rcout << }\StringTok{"b2: "}\NormalTok{ << b2 << }\StringTok{'}\SpecialCharTok{\textbackslash{}n}\StringTok{'}\NormalTok{;}
\NormalTok{  Rcout << }\StringTok{"b3: "}\NormalTok{ << b3 << }\StringTok{'}\SpecialCharTok{\textbackslash{}n}\StringTok{'}\NormalTok{;}
  \ControlFlowTok{return}\NormalTok{ List::create(lv1, lv2, lv3, b1, b2, b3);}
\NormalTok{\}}
\end{Highlighting}
\end{Shaded}

在R中的运行结果为:

\begin{Shaded}
\begin{Highlighting}[]
\OperatorTok{>}\StringTok{ }\KeywordTok{sourceCpp}\NormalTok{(}\StringTok{'test.cpp'}\NormalTok{)}

\OperatorTok{>}\StringTok{ }\NormalTok{test_list <-}\StringTok{ }\KeywordTok{rcpp_logical_03}\NormalTok{()}
\KeywordTok{all}\NormalTok{( v1 ) is }\OtherTok{NA}
\NormalTok{lv1}\OperatorTok{:}\StringTok{ }\OperatorTok{-}\DecValTok{2147483648}
\NormalTok{lv2}\OperatorTok{:}\StringTok{ }\DecValTok{0}
\NormalTok{lv3}\OperatorTok{:}\StringTok{ }\DecValTok{1}
\NormalTok{b1}\OperatorTok{:}\StringTok{ }\DecValTok{0}
\NormalTok{b2}\OperatorTok{:}\StringTok{ }\DecValTok{0}
\NormalTok{b3}\OperatorTok{:}\StringTok{ }\DecValTok{1}
\end{Highlighting}
\end{Shaded}

需要注意的是,在@ref(\#LogicalVector-elements)中提到过,\texttt{NA}的值为-2147483648,与打印的\texttt{lv1}信息一致。

\subsection{ifelse()}\label{ifelse}

\texttt{ifelse\ (v,\ x1,\ x2)}
接收逻辑向量\texttt{v},如果\texttt{v}中的某元素为\texttt{TRUE},那么返回\texttt{x1}中对应位置的元素,如果为\texttt{FLASE},那么返回x2中对应位置的元素。尽管\texttt{x1}和\texttt{x2}可以是标量或者向量,但如果是向量,两者的长度必须与\texttt{v}的长度一致。

\begin{Shaded}
\begin{Highlighting}[]
\CommentTok{// [[Rcpp::export]]}
\DataTypeTok{int}\NormalTok{ rcpp_logical_02(NumericVector v1, NumericVector v2)\{}
  
  \CommentTok{//向量元素个数}
  \DataTypeTok{int}\NormalTok{ n = v1.length();}
  \CommentTok{// 情况1:x1 和 x2是标量的情况}
\NormalTok{  IntegerVector res1     = ifelse( v1>v2, }\DecValTok{1}\NormalTok{, }\DecValTok{0}\NormalTok{);}
\NormalTok{  NumericVector res2     = ifelse( v1>v2, }\FloatTok{1.0}\NormalTok{, }\FloatTok{0.0}\NormalTok{);}
  
  \CommentTok{//CharacterVector res3 = ifelse( v1>v2, "T", "F"); // 不支持此种写法}
  \CommentTok{//ifelse() 不支持字符串标量,为了得到和R一样的结果}
  \CommentTok{// 我们需要使用字符串向量,该向量所有元素相同 }
\NormalTok{  CharacterVector chr_v1 = rep(CharacterVector(}\StringTok{"T"}\NormalTok{), n);}
\NormalTok{  CharacterVector chr_v2 = rep(CharacterVector(}\StringTok{"F"}\NormalTok{), n);}
\NormalTok{  CharacterVector res3   = ifelse( v1>v2, chr_v1, chr_v2);}
\NormalTok{  Rcout <<}\StringTok{"case1: both x1 and x2 are scalar"}\NormalTok{<<}\StringTok{'}\SpecialCharTok{\textbackslash{}n}\StringTok{'}\NormalTok{;}
\NormalTok{  Rcout << }\StringTok{"}\SpecialCharTok{\textbackslash{}t}\StringTok{ v1 > v2 "}\NormalTok{ << res1 <<}\StringTok{'}\SpecialCharTok{\textbackslash{}n}\StringTok{'}\NormalTok{;}
\NormalTok{  Rcout << }\StringTok{"}\SpecialCharTok{\textbackslash{}t}\StringTok{ v1 > v2 "}\NormalTok{ << res2 <<}\StringTok{'}\SpecialCharTok{\textbackslash{}n}\StringTok{'}\NormalTok{;}
\NormalTok{  Rcout << }\StringTok{"}\SpecialCharTok{\textbackslash{}t}\StringTok{ v1 > v2 "}\NormalTok{ << res3 <<}\StringTok{'}\SpecialCharTok{\textbackslash{}n}\StringTok{'}\NormalTok{;}
  
  \CommentTok{//情况2,x1是向量,x2是标量}
\NormalTok{  IntegerVector int_v1, int_v2;}
\NormalTok{  int_v1 = rep(}\DecValTok{1}\NormalTok{,n);}
\NormalTok{  int_v2 = rep(}\DecValTok{0}\NormalTok{,n);}
\NormalTok{  NumericVector num_v1, num_v2;}
\NormalTok{  num_v1 = rep(}\DecValTok{1}\NormalTok{.,n);}
\NormalTok{  num_v2 = rep(}\DecValTok{0}\NormalTok{.,n);}
  
\NormalTok{  IntegerVector   res4 = ifelse( v1>v2, int_v1, }\DecValTok{0}\NormalTok{);}
\NormalTok{  NumericVector   res5 = ifelse( v1>v2, num_v1, }\FloatTok{0.0}\NormalTok{);}
\NormalTok{  CharacterVector res6 = ifelse( v1>v2, chr_v1, Rf_mkChar(}\StringTok{"F"}\NormalTok{)); }\CommentTok{// Note}
  
\NormalTok{  Rcout <<}\StringTok{"case2: x1 and x2 are vector and scalar"}\NormalTok{<<}\StringTok{'}\SpecialCharTok{\textbackslash{}n}\StringTok{'}\NormalTok{;}
\NormalTok{  Rcout << }\StringTok{"}\SpecialCharTok{\textbackslash{}t}\StringTok{ v1 > v2 "}\NormalTok{ << res4 <<}\StringTok{'}\SpecialCharTok{\textbackslash{}n}\StringTok{'}\NormalTok{;}
\NormalTok{  Rcout << }\StringTok{"}\SpecialCharTok{\textbackslash{}t}\StringTok{ v1 > v2 "}\NormalTok{ << res5 <<}\StringTok{'}\SpecialCharTok{\textbackslash{}n}\StringTok{'}\NormalTok{;}
\NormalTok{  Rcout << }\StringTok{"}\SpecialCharTok{\textbackslash{}t}\StringTok{ v1 > v2 "}\NormalTok{ << res6 <<}\StringTok{'}\SpecialCharTok{\textbackslash{}n}\StringTok{'}\NormalTok{;}
  
  \CommentTok{//情况3,x1和x2均为向量}
\NormalTok{  IntegerVector   res7 = ifelse( v1>v2, int_v1, int_v2);}
\NormalTok{  NumericVector   res8 = ifelse( v1>v2, num_v1, num_v2);}
\NormalTok{  CharacterVector res9 = ifelse( v1>v2, chr_v1, chr_v2);}
  
\NormalTok{  Rcout <<}\StringTok{"case3: both x1 and x2 are vector"}\NormalTok{<<}\StringTok{'}\SpecialCharTok{\textbackslash{}n}\StringTok{'}\NormalTok{;}
\NormalTok{  Rcout << }\StringTok{"}\SpecialCharTok{\textbackslash{}t}\StringTok{ v1 > v2 "}\NormalTok{ << res7 <<}\StringTok{'}\SpecialCharTok{\textbackslash{}n}\StringTok{'}\NormalTok{;}
\NormalTok{  Rcout << }\StringTok{"}\SpecialCharTok{\textbackslash{}t}\StringTok{ v1 > v2 "}\NormalTok{ << res8 <<}\StringTok{'}\SpecialCharTok{\textbackslash{}n}\StringTok{'}\NormalTok{;}
\NormalTok{  Rcout << }\StringTok{"}\SpecialCharTok{\textbackslash{}t}\StringTok{ v1 > v2 "}\NormalTok{ << res9 <<}\StringTok{'}\SpecialCharTok{\textbackslash{}n}\StringTok{'}\NormalTok{;}
  
  \ControlFlowTok{return} \DecValTok{0}\NormalTok{;}
\NormalTok{\}}
\end{Highlighting}
\end{Shaded}

Note:
\texttt{Rf\_mkChar\ ()}函数作用为将C语言中的字符串(\texttt{char*})转为\texttt{CHARSXP}(\texttt{CharacterVector}中的元素类型)。

在R中运行结果为:

\begin{Shaded}
\begin{Highlighting}[]
\OperatorTok{>}\StringTok{ }\KeywordTok{sourceCpp}\NormalTok{(}\StringTok{'test.cpp'}\NormalTok{)}

\OperatorTok{>}\StringTok{ }\NormalTok{tmp <-}\StringTok{ }\KeywordTok{rcpp_logical_02}\NormalTok{(}\DecValTok{1}\OperatorTok{:}\DecValTok{4}\NormalTok{,}\DecValTok{4}\OperatorTok{:}\DecValTok{1}\NormalTok{)}
\NormalTok{case1}\OperatorTok{:}\StringTok{ }\NormalTok{both x1 and x2 are scalar}
\NormalTok{     v1 }\OperatorTok{>}\StringTok{ }\NormalTok{v2 }\DecValTok{0} \DecValTok{0} \DecValTok{1} \DecValTok{1}
\NormalTok{     v1 }\OperatorTok{>}\StringTok{ }\NormalTok{v2 }\DecValTok{0} \DecValTok{0} \DecValTok{1} \DecValTok{1}
\NormalTok{     v1 }\OperatorTok{>}\StringTok{ }\NormalTok{v2 }\StringTok{"F"} \StringTok{"F"} \StringTok{"T"} \StringTok{"T"}
\NormalTok{case2}\OperatorTok{:}\StringTok{ }\NormalTok{x1 and x2 are vector and scalar}
\NormalTok{     v1 }\OperatorTok{>}\StringTok{ }\NormalTok{v2 }\DecValTok{0} \DecValTok{0} \DecValTok{1} \DecValTok{1}
\NormalTok{     v1 }\OperatorTok{>}\StringTok{ }\NormalTok{v2 }\DecValTok{0} \DecValTok{0} \DecValTok{1} \DecValTok{1}
\NormalTok{     v1 }\OperatorTok{>}\StringTok{ }\NormalTok{v2 }\StringTok{"F"} \StringTok{"F"} \StringTok{"T"} \StringTok{"T"}
\NormalTok{case3}\OperatorTok{:}\StringTok{ }\NormalTok{both x1 and x2 are vector}
\NormalTok{     v1 }\OperatorTok{>}\StringTok{ }\NormalTok{v2 }\DecValTok{0} \DecValTok{0} \DecValTok{1} \DecValTok{1}
\NormalTok{     v1 }\OperatorTok{>}\StringTok{ }\NormalTok{v2 }\DecValTok{0} \DecValTok{0} \DecValTok{1} \DecValTok{1}
\NormalTok{     v1 }\OperatorTok{>}\StringTok{ }\NormalTok{v2 }\StringTok{"F"} \StringTok{"F"} \StringTok{"T"} \StringTok{"T"}
\end{Highlighting}
\end{Shaded}

\section{LogicalVector元素的估值}\label{Evaluation-LogicalVector}

\texttt{LogicalVector}的元素值不应当被用作\texttt{if}语句的条件表达式。因为,C++中\texttt{if}语句将条件表达式评估为\texttt{bool}型。而\texttt{bool}型把所有非零的值均评估为\texttt{true},因此,\texttt{LogicalVector}中的\texttt{NA}(\texttt{NA\_LOGICAL})也会被认为是\texttt{true}。

下面的代码示例展示了\texttt{if}语句是如何评估\texttt{LogicalVector}的元素值。

\begin{Shaded}
\begin{Highlighting}[]
\CommentTok{// [[Rcpp::plugins(cpp11)]]}
\CommentTok{// [[Rcpp::export]]}
\NormalTok{LogicalVector rcpp_logical()\{}
  \CommentTok{// 构建一个包含NA值得整型向量}
\NormalTok{  IntegerVector x = \{}\DecValTok{1}\NormalTok{,}\DecValTok{2}\NormalTok{,}\DecValTok{3}\NormalTok{,}\DecValTok{4}\NormalTok{,NA_INTEGER\};}
  \CommentTok{// 比较运算的结果是逻辑向量}
\NormalTok{  LogicalVector v = (x >= }\DecValTok{3}\NormalTok{);}
  \CommentTok{//如果将逻辑向量的元素直接用于if语句中,NA_LOGICAL会被认为是TRUE}
  \ControlFlowTok{for}\NormalTok{(}\DataTypeTok{int}\NormalTok{ i=}\DecValTok{0}\NormalTok{; i<v.size();++i) \{}
    \ControlFlowTok{if}\NormalTok{(v[i]) Rprintf(}\StringTok{"v[}\SpecialCharTok{%i}\StringTok{]:}\SpecialCharTok{%i}\StringTok{ is evaluated as true.}\SpecialCharTok{\textbackslash{}n}\StringTok{"}\NormalTok{,i,v[i]);}
    \ControlFlowTok{else}\NormalTok{ Rprintf(}\StringTok{"v[}\SpecialCharTok{%i}\StringTok{]:}\SpecialCharTok{%i}\StringTok{ is evaluated as false.}\SpecialCharTok{\textbackslash{}n}\StringTok{"}\NormalTok{,i,v[i]);}
\NormalTok{  \}}
  \CommentTok{// 评估逻辑向量的元素}
  \ControlFlowTok{for}\NormalTok{(}\DataTypeTok{int}\NormalTok{ i=}\DecValTok{0}\NormalTok{; i<v.size();++i) \{}
    \ControlFlowTok{if}\NormalTok{(v[i]==TRUE) Rprintf(}\StringTok{"v[}\SpecialCharTok{%i}\StringTok{] is TRUE.}\SpecialCharTok{\textbackslash{}n}\StringTok{"}\NormalTok{,i);}
    \ControlFlowTok{else} \ControlFlowTok{if}\NormalTok{ (v[i]==FALSE) Rprintf(}\StringTok{"v[}\SpecialCharTok{%i}\StringTok{] is FALSE.}\SpecialCharTok{\textbackslash{}n}\StringTok{"}\NormalTok{,i);}
    \ControlFlowTok{else} \ControlFlowTok{if}\NormalTok{ (v[i]==NA_LOGICAL) Rprintf(}\StringTok{"v[}\SpecialCharTok{%i}\StringTok{] is NA.}\SpecialCharTok{\textbackslash{}n}\StringTok{"}\NormalTok{,i);}
    \ControlFlowTok{else}\NormalTok{ Rcout << }\StringTok{"v["}\NormalTok{ << i << }\StringTok{"] is not 1}\SpecialCharTok{\textbackslash{}n}\StringTok{"}\NormalTok{;}
\NormalTok{  \}}
  \CommentTok{// 打印TRUE, FALSE 和 NA_LOGICAL的值}
\NormalTok{  Rcout << }\StringTok{"TRUE "}\NormalTok{ << TRUE << }\StringTok{"}\SpecialCharTok{\textbackslash{}n}\StringTok{"}\NormalTok{;}
\NormalTok{  Rcout << }\StringTok{"FALSE "}\NormalTok{ << FALSE << }\StringTok{"}\SpecialCharTok{\textbackslash{}n}\StringTok{"}\NormalTok{;}
\NormalTok{  Rcout << }\StringTok{"NA_LOGICAL "}\NormalTok{ << NA_LOGICAL << }\StringTok{"}\SpecialCharTok{\textbackslash{}n}\StringTok{"}\NormalTok{;}
  \ControlFlowTok{return}\NormalTok{ v;}
\NormalTok{\}}
\end{Highlighting}
\end{Shaded}

需要注意的是,在原始代码上,需要加入\texttt{//\ {[}{[}Rcpp::plugins(cpp11){]}{]}}语句启动C++11的特性,否则代码会因为初始化语句而报错。

执行结果为:

\begin{Shaded}
\begin{Highlighting}[]
\OperatorTok{>}\StringTok{ }\KeywordTok{sourceCpp}\NormalTok{(}\StringTok{'test.cpp'}\NormalTok{)}

\OperatorTok{>}\StringTok{ }\KeywordTok{rcpp_logical}\NormalTok{()}
\NormalTok{v[}\DecValTok{0}\NormalTok{]}\OperatorTok{:}\DecValTok{0}\NormalTok{ is evaluated as false.}
\NormalTok{v[}\DecValTok{1}\NormalTok{]}\OperatorTok{:}\DecValTok{0}\NormalTok{ is evaluated as false.}
\NormalTok{v[}\DecValTok{2}\NormalTok{]}\OperatorTok{:}\DecValTok{1}\NormalTok{ is evaluated as true.}
\NormalTok{v[}\DecValTok{3}\NormalTok{]}\OperatorTok{:}\DecValTok{1}\NormalTok{ is evaluated as true.}
\NormalTok{v[}\DecValTok{4}\NormalTok{]}\OperatorTok{:-}\DecValTok{2147483648}\NormalTok{ is evaluated as true.}
\NormalTok{v[}\DecValTok{0}\NormalTok{] is FALSE.}
\NormalTok{v[}\DecValTok{1}\NormalTok{] is FALSE.}
\NormalTok{v[}\DecValTok{2}\NormalTok{] is TRUE.}
\NormalTok{v[}\DecValTok{3}\NormalTok{] is TRUE.}
\NormalTok{v[}\DecValTok{4}\NormalTok{] is NA.}
\OtherTok{TRUE} \DecValTok{1}
\OtherTok{FALSE} \DecValTok{0}
\NormalTok{NA_LOGICAL }\OperatorTok{-}\DecValTok{2147483648}
\NormalTok{[}\DecValTok{1}\NormalTok{] }\OtherTok{FALSE} \OtherTok{FALSE}  \OtherTok{TRUE}  \OtherTok{TRUE}    \OtherTok{NA}
\end{Highlighting}
\end{Shaded}

\bibliography{book.bib,packages.bib}

\backmatter
\printindex

\end{document}
